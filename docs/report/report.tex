\documentclass[a4paper,10pt]{article}
\usepackage[utf8]{inputenc}

\usepackage{pattern}
\usepackage{listings}
\usepackage{enumitem}
\usepackage{todo}
\usepackage{appendix}

% Has to be the last package:
\usepackage [hidelinks] {hyperref}


\newcommand{\dir} [1] [] {#1} 
\newcommand{\todoref}{\todo{ref}}

% Title Page
\title{Concurrency Patterns in SCOOP}
\author{Roman Schmocker}


\begin{document}
\maketitle

\begin{abstract}

\end{abstract}

\tableofcontents

\section{Introduction}

Due to the advent of multicore processors, concurrent programming has become an important part in software engineering.
Dealing with parallelism isn't easy however.
There are many pitfalls, such as race conditions and deadlocks.

In practice programmers have learned to avoid tricky concurrency problems with the use of some well-known patterns \todo{is it ok to take introduction from project plan almost word for word?}.
These patterns are often shipped with the standard library of the language, such that users rarely have to implement them.

The Eiffel language \todo{ref} has a new concurrency extension called SCOOP, which stands for Simple Concurrent Object-Oriented Programming.
SCOOP simplifies concurrent programming a lot and eliminates one source of errors completely, namely race conditions \todo{ref}.
However, there is little experience on how to implement popular concurrency patterns, such as a worker pool, in SCOOP.

The goal of this thesis is therefore to implement a standard library for concurrency patterns in Eiffel.

\subsection{Overview}

Section \ref{sec:pattern_overview} introduces a list of concurrency patterns which we found and categorized by studying literature and the standard libraries.
A brief introduction to the SCOOP model is given in Section \ref{sec:scoop-model}.
Section \ref{sec:scoop-challenges} describes some challenges when programming in SCOOP and how to solve them.
The latter two sections may be interesting for programmers with experience in threaded programming, but who wish to learn SCOOP.

The focus of Section \ref{sec:library} describes the goals and concepts of the concurrency patterns library.
It also provides a brief overview over the available modules, and which patterns are implemented by which modules.

A detailed explanation over the individual modules is provided by Section \ref{sec:modules}.
Finally, Section \ref{sec:evaluation} provides a small performance evaluation of the library.

\section {Pattern overview}
\label{sec:pattern_overview}
% Overview of all patterns, maybe tabular

\section {Pattern overview}
\label{sec:pattern_overview}
% Overview of all patterns

\subsection{Data-centric patterns}

\pattern [name={Producer / Consumer}
  ,label={P/C}
  ,category={Program structuring}
  ,intent={Provide a synchronized shared buffer. Producer threads put items into the buffer, and consumers remove items.}
  ,applicability={When participants should not know each other. Also applicable if there's no one-to-one relation between producers and consumers or when buffering is desired.}
  ,status={Implemented library component}
  ,example={A logger service where many producers submit log messages to a buffer and a single consumer writes them to a file.}
  ,knownapps={Very widely used. E.g. logging, input processing, buffering web server requests.}
  ,relation={The Worker Pool \patternref{WP} uses this pattern for its task queue. Pipeline \patternref{PL} and Dataflow Networks \patternref {DFN} are chained Producer / Consumer instances.}
 ,references={\cite[p. 87]{book:java-concurrency}, \cite[p. 53]{book:ppl}}
]

\pattern [name={Pipeline}
  ,label={PL}
  ,category={Program structuring}
  ,intent={Process data in several independent stages.}
  ,applicability={When the input consist of a stream of data where several processing steps need to be performed.}
  ,status={Possible library component}
  ,example={An emailing system that applies a spam filter, database logging, and a virus scan to each incoming email.}
  ,knownapps={Messaging systems, multimedia streaming (receive - decode - display)}
  ,relation={The Producer / Consumer pattern \patternref{P/C} is used between two stages. Pipeline is a special form of Dataflow Network \patternref{DFN}.}
 ,references={\cite[p. 305]{book:cpj}, \cite[p. 253]{book:spp}, \cite[p. 53]{book:ppl}}
]

\pattern [name={Dataflow Network}
  ,label={DFN}
  ,category={Program structuring}
  ,intent={Process data in independent stages, with the option to branch and merge data streams.}
  ,applicability={When the input consists of a stream of data which allows for parallel processing.}
  ,status={Possible library component}
  ,example={A video player application that internally has a file decoder stage, which splits the input in an audio and video part for further processing.}
  ,knownapps={The borealis engine \cite{web:borealis}.}
  ,relation={The pattern is related to Pipeline \patternref{PL}. In Dataflow Network however data can be split and forwarded to two different stages and maybe merged again later.}
 ,references={\cite[p. 305]{book:cpj}, \cite[p. 261]{book:spp}}
]

\pattern [name={Exchanger}
  ,label={EXC}
  ,category={Program structurings}
  ,intent={Exchange two objects between two threads atomically.}
  ,applicability={When synchronization and atomicity is required.}
  ,status={Possible library component.}
  ,example={A logger with two buffers: One is used by clients to submit messages, the other is used by the logger to write messages. When the latter is empty and the former is full the exchange happens.}
  ,knownapps={ - }
  ,relation={Similar to Synchronous Message Passing, except that data passes in both directions.}
  ,references={\cite[p. 101]{book:java-concurrency}, \cite[p. 231]{book:cpj}}
]

\subsection{Task-centric patterns}

\pattern [name={Worker Pool}
  ,label={WP}
  ,category={Performance}
  ,intent={Avoid expensive thread creation by providing a set of threads that can execute arbitrary operations.}
  ,applicability={When there are a lot of small tasks that may be executed in parallel.}
  ,status={Implemented library component}
  ,example={A set of HTTP request handlers in a web server.}
  ,knownapps={Often used in server applications, e.g. databases, HTTP servers, web services.}
  ,relation={Producer / Consumer \patternref{P/C} is used to pass along task objects. Worker Pool is usually an implementation of the Executor Framework \patternref{EF}.}
 ,references={\cite[p. 117]{book:java-concurrency}, \cite[p. 167]{book:java-concurrency}, \cite[p. 290]{book:cpj},  \cite[p. 61]{book:ppl}, \cite{paper:tpl}}
]

\pattern [ name={Future}
  ,label={FUT}
  ,category={Program structuring, Performance}
  ,intent={Run a task asynchronously and fetch the result later.}
  ,applicability={When a computation may be run in parallel, but creating an extra thread is too expensive.}
  ,status={Implemented library component}
  ,example={A web browser which starts download tasks for image files in parallel to rendering an HTML file.}
  ,knownapps={In UI programming for long-running background tasks, or parallelization of numerical computations.}
  ,relation={Futures may be backed by a Worker Pool \patternref{WP} that execute them.}
  ,references={\cite[p. 125]{book:java-concurrency}, \cite[p. 332]{book:cpj}, \cite[p. 36]{book:ppl}, \cite{paper:tpl}}
  ,comment={The \emph{wait by necessity} semantics of SCOOP \cite{web:scoop} also correspond to the Future pattern.}
]

\pattern [name={Executor Framework}
  ,label={EF}
  ,category={Program structuring}
  ,intent={Split task submission from task execution.}
  ,applicability={When the task execution strategy should be flexible, e.g. using a worker pool or creating a new thread per task.}
  ,status={Implemented library component}
  ,example={The Java Executor interface, where descendants can decide wether a submitted Runnable object is executed in the current thread, in a new thread, or by a worker pool.}
  ,knownapps={Java Executor interface, Microsoft Task Parallel Library}
  ,relation={The Worker Pool \patternref{WP} is an implementation of the Executor Framework.}
  ,references={\cite[p. 117]{book:java-concurrency}, \cite[p. 289]{book:cpj}}
]

\pattern [name={Timer: Periodic}
  ,label={TP}
  ,category={Program structuring}
  ,intent={Apply an operation repeatedly in regular intervals.}
  ,applicability={When an operation, which can be executed in parallel to the application's main thread, needs to be applied repeatedly.}
  ,status={Implemented library component}
  ,example={An email client that checks for new messages every five seconds.}
  ,knownapps={Message polling, buffer flushes, background log writes, heartbeat messages, cron jobs.}
  ,relation={Similar to Active Object \patternref{AO}, but it schedules just one operation repeatedly.}
  ,references={\cite[p. 123]{book:java-concurrency}, \cite[p. 298]{book:cpj}}
]

\pattern [ name={Timer: Invoke Later}
  ,label={TIL}
  ,category={Program structuring}
  ,intent={Invoke a certain operation at a later point in time.}
  ,applicability={When it is necessary to wait a bit before executing an operation.}
  ,status={Implemented library component}
  ,example={Send an email after a delay of one minute, during which the user can still press a cancel button.}
  ,knownapps={``Grace periods'' to cancel actions, robotics control, alarm clocks.}
  ,relation={ - }
  ,references={\cite[p. 123]{book:java-concurrency}, \cite[p. 297]{book:cpj}}
]

\subsection{I/O patterns}

\pattern [name={Half-Sync / Half-Async}
  ,label={HS/HA}
  ,category={Program structuring}
  ,intent={Simplify asynchronous event handling. A thread or an interrupt handler listens for incoming messages and puts them in a synchronized queue. Worker threads then retrieve and handle the messages.}
  ,applicability={When the application must react to several event sources at the same time.}
  ,status={not covered}
  ,example={The network stack in most UNIX system is implemented like this. A network socket is the ``queue'' which gets filled by interrupt handlers. Application threads take care of handling the data.}
  ,knownapps={Network sockets, web servers.}
  ,relation={ - }
  ,references={\cite[p. 423]{book:posa2}}
]

\pattern [name={Leader / Followers}
  ,label={L/F}
  ,category={Performance, Program structuring}
  ,intent={Reduce synchronization overhead when using a thread pool to handle requests on I/O sockets. A leader thread receives a request, promotes the next leader, and then handles the request.}
  ,applicability={When there are hundreds of I/O sockets.}
  ,status={not covered}
  ,example={A web server for a high volume website serving thousands of connections at the same time.}
  ,knownapps={Online Transaction Processing (OLTP) applications.}
  ,relation={Compared to Half-Sync / Half-Async \patternref{HS/HA} it avoids the synchronization overhead of a shared queue.}
  ,references={\cite[p. 447]{book:posa2}, \cite{paper:leader-follower}}
]

\pattern [name={Disruptor}
  ,label={DIS}
  ,category={Performance, Program structuring}
  ,intent={Provide a high-performance ring buffer with a single producer and multiple readers, each assigned to a thread. Readers can have dependencies to other readers and change buffer entries.}
  ,applicability={When very high throughput in an I/O application is required.}
  ,status={not covered}
  ,example={An OLTP system where the producer listens on a socket for new requests. Then there's a reader for each of the following tasks: logging, unmarshalling, request handling.}
  ,knownapps={LMAX Exchange uses this pattern for their trading platform.}
  ,relation={Similar to Half-Synch/Half-Asynch \patternref{HS/HA}, but buffer entries may be modified in place and accessed by several threads.}
 ,references={\cite{paper:disruptor}}
]

\subsection{Miscellaneous patterns}

\pattern [name={Active Object}
  ,label={AO}
  ,category={Program structuring}
  ,intent={Pair an object with its own thread. Clients access the active object through a proxy which transforms feature calls to asynchronous messages. 
  The active object runs a main loop where it schedules requests from clients and runs its own code.}
  ,applicability={When access to a shared resource can be guarded by an object, or when an object should execute its own main loop.}
  ,status={Implemented language mechanism. Implemented library component.}
  ,example={A logging service may be implemented as an active object. Clients call \lstinline!log ("something")! on the proxy which forwards the message to the active object.}
  ,knownapps={The Java Timer class is implemented as an active object. SCOOP separate calls correspond to feature invocation on an active object.}
  ,relation={The Future pattern \patternref{FUT} is usually used for active object functions that return a result.}
  ,references={\cite[p. 369]{book:posa2}, \cite[p. 367]{book:cpj}}
]

\pattern [ name={Thread-local storage}
  ,label={TLS}
  ,category={Program structuring}
  ,intent={Provide private heap data for each thread.}
  ,applicability={When multiple threads run the same code, but each one needs a different set of data, or when the synchronization overhead for shared heap objects is undesirable.}
  ,status={Implemented language mechanism}
  ,example={Store the last exception raised in the current thread.}
 ,knownapps={Java and C\# both have a class \lstinline!ThreadLocal<T>!.}
 ,relation={ - }
 ,references={\cite[p. 475]{book:posa2}, \cite[p. 45]{book:java-concurrency}, \cite[p. 105 ]{book:cpj}, \cite[p. 53]{book:ppl}}
 ,comment={Native support in SCOOP: Use \lstinline!once("THREAD")! and a non-separate return type.}
]

\pattern [name={Publish / Subscribe}
  ,label={P/S}
  ,category={Program structuring}
  ,intent={Provide a hook to subscribe to events. In the concurrent context there's often an intermediate broker which receives events from a publisher and forwards them to all subscribers.}
  ,applicability={When the publisher doesn't need to know the subscribers, and vice versa with the broker solution.}
  ,status={Implemented library component}
  ,example={A GUI button has an event ``clicked''. The application logic can subscribe to it with a handler function.}
  ,knownapps={Event driven programming, GUI frameworks like Java Swing or EiffelVision.}
  ,relation={Similar to the Observer pattern by Gamma et al.\cite[p. 293]{book:design-patterns}, but events may come with arguments. The Eiffel agent mechanism may be used for Publish / Subscribe.}
 ,references={ - }
]

\pattern [name={Transactions}
  ,label={TRA}
  ,category={Program structuring}
  ,intent={Avoid a deadlock by reserving a set of objects one at a time. Abort if an object is already reserved by another thread.}
  ,applicability={When multiple operations need to be locked and no proper locking order can be established.}
  ,status={Possible library component}
  ,example={A banking application where multiple threads apply various operations on a set of bank accounts.}
  ,knownapps={Two-phase locking in database systems.}
  ,relation={ - }
 ,references={\cite[p. 249]{book:cpj}}
]

\subsection{SCOOP patterns}

\pattern [ name={Import}
  ,label={IMP}
  ,category={Language limitation}
  ,intent={Copy an object structure from a separate processor to the local processor.}
  ,applicability={When it's cheaper to clone the object instead of placing it on its own processor.}
  ,status={Implemented library component; future language mechanism}
  ,example={Copy the HTTP request string from the network socket listener to a request handler, such that the listener can continue.}
  ,knownapps={The library developed in this thesis makes heavy use of this pattern.}
  ,relation={ - }
 ,references={\cite[p. 106]{Nienaltowski07}}
]

\pattern [ name={Asynchronous Self-Call}
  ,label={ASC}
  ,category={Program structuring}
  ,intent={Execute the body of a main loop and then ask another processor to call back the loop body.}
  ,applicability={When a processor is running its own code, but others need to access data on it from time to time.}
  ,status={Implemented library component}
  ,example={A network socket listener that may be stopped by another processor.}
  ,knownapps={The Timer: Periodic \patternref{TP} implementation and the echo server example (see Appendix \ref{sec:echo-server}) use asynchronous self-calls.}
  ,relation={Similar to the Active Object pattern \patternref{AO}, but the Asynchronous Self-Call pattern lets other processors manipulate its data directly.}
  ,references={\cite[p. 217]{Nienaltowski07}}
]

\pattern [ name={Separate Proxy}
  ,label={SP}
  ,category={Program structuring}
  ,intent={Simplify access to a separate object by providing a processor-local proxy.}
  ,applicability={When a class is reusable (i.e. library code) and usually placed on a separate processor.}
  ,status={Guideline}
  ,example={A shared queue which gets accessed by several threads. Each thread creates a processor-local proxy to avoid having to deal with a separate reference.}
  ,knownapps={Most classes in the library have a Separate Proxy.}
  ,relation={The Separate Proxy is a special version of the proxy pattern described by Gamma et al. \cite[p. 207]{book:design-patterns}.}
  ,references={ - }
]

\pattern [name={Full Asynchrony}
  ,label={FA}
  ,category={Language limitation}
  ,intent={Perform an operation completely asynchronously.}
  ,applicability={When an operation can be run in parallel and there's no need to wait for a result.}
  ,status={Future language mechanism}
  ,example={A logger service where clients just want to send a log message without having to wait.}
  ,knownapps={A workaround is described in \cite[p. 215]{Nienaltowski07}, but it is currently broken in SCOOP.}
  ,relation={ - }
  ,references={\cite[p. 215]{Nienaltowski07}}
  , comment={Will be supported natively in the new runtime developed by Scott West \cite{thesis:scottwest}.}
]

\pattern [name={Universal Call}
  ,label={UC}
  ,category={Language limitation}
  ,intent={Provide a universal enclosing routine to perform a single call on a separate object.}
  ,applicability={When it doesn't matter if separate calls are interleaved with calls from other processors.}
  ,status={Designed language mechanism}
  ,example={A shared queue where producers only insert items.}
  ,knownapps={An implementation is described in \cite[p. 213]{Nienaltowski07}, but it is currently broken in SCOOP.}
  ,relation={The Separate Proxy \patternref{SP} is a workaround for the missing universal call.}
  ,references={\cite[p. 213]{Nienaltowski07}}
  ,comment={The new language mechanism will probably be a statement like:\newline \lstinline!separate a as l_a then l_a.do_something end!}
]

\subsection{Synchronization primitives}


\syncpattern [name={Atomic Operations}
  ,category={Synchronization primitive}
  ,intent={Avoid the use of locks by using hardware-supported atomic operations.}
  ,status={not covered}
  ,example={A lock-free queue using CompareAndSwap.}
  ,knownapps={Low-level primitive which is used to implement lock-free data structures or other synchronization primitives.}
  ,references={\cite[p. 319]{book:java-concurrency}, \cite[p. 140]{book:cpj}}
]
  
\syncpattern [name={Locks}
  ,category={Synchronization primitive}
  ,intent={An object where only one thread at a time succeeds in calling \lstinline!lock!, and others have to wait.}
  ,status={Possible library component}
  ,example={Provide exclusive access on a certain section of code.}
  ,knownapps={Low-level primitive which is often used to implement other synchronization primitives.}
  ,references={\cite[p. 277]{book:java-concurrency}, \cite[p. 148]{book:cpj}}
]

\syncpattern [name={Try Lock}
  ,category={Synchronization primitive}
  ,intent={Try to acquire a lock with the option to back off after a certain amount of time.}
  ,status={Possible library component}
  ,example={Database transactions may get aborted due to a timeout if they can't lock a resource after a certain amount of time.}
  ,knownapps={Applications with real-time requirements.}
  ,references={\cite[p. 277]{book:java-concurrency}, \cite[p. 148]{book:cpj}}
]

\syncpattern [name={Read / Write lock}
  ,category={Synchronization primitive}
  ,intent={Allow multiple concurrent readers but provide exclusive access to a writer.}
  ,status={Language limitation}
  ,example={An array with frequent concurrent reads can make use of a read / write lock.}
  ,knownapps={Shared, read-mostly data structures.}
  ,references={\cite[p. 286]{book:java-concurrency}, \cite[p. 157]{book:cpj}}
]

\syncpattern [name={Semaphore}
  ,category={Synchronization primitive}
  ,intent={Make sure that only a certain amount of threads can execute a section of code.}
  ,status={Possible library component}
  ,example={The dining philosophers pattern, where at most (N-1) philosophers can eat.}
  ,knownapps={Can be used to implement other synchronization primitives.}
  ,references={\cite[p. 98]{book:java-concurrency}, \cite[p. 220]{book:cpj}}
]

\syncpattern [ name={Single Exclusive Access}
  ,category={Synchronization Primitive}
  ,intent={Make sure that at most one thread has access to exactly one shared object or resource.}
  ,status={Implemented language mechanism}
  ,example={A counter variable that shold only be incremented by one thread at a time to avoid lost updates.}
  ,knownapps={The Java \lstinline!synchronized! and C\# \lstinline!lock! statements implement single exclusive access for sections of code.}
  ,references={\cite[p. 25]{book:java-concurrency}, \cite[p. 76]{book:cpj}}
  ]

\syncpattern [ name={Multiple Exclusive Access}
  ,category={Synchronization Primitive}
  ,intent={Make sure that at most one thread has access to several shared objects or resources.}
  ,status={Implemented language mechanism}
  ,example={A money transfer between two bank accounts.}
  ,knownapps={Databases can provide exclusive access over all data items previously used in the same transaction.}
%   ,comment={It is possible to use nested single exclusive access to provide multiple exclusive access, but special care has to be taken with deadlocks.}
  ,references={ - }
  ]

\syncpattern [name={Barrier}
  ,category={Synchronization primitive}
  ,intent={Provide a synchronization point where several threads have to meet before continuing.}
  ,status={Possible library component}
  ,example={If the computation of a matrix multiplication is divided among threads, a barrier can be used to make sure that all threads finish before the result is used.}
  ,knownapps={Parallel matrix operations, parallel loop processing.}
  ,references={\cite[p. 99]{book:java-concurrency}, \cite[p. 362]{book:cpj}}
]

\syncpattern [name={Monitor}
  ,category={Synchronization primitive}
  ,intent={Ensure that only one thread has access to an object. The thread may also wait for a condition to become true.}
  ,status={Implemented language mechanism}
  ,example={A shared buffer with conditions \lstinline!is_empty! and \lstinline!is_full!.}
  ,knownapps={Java with a combination of \lstinline!synchronized!, \lstinline!wait()! and \lstinline!notifiyAll()!}
  ,comment={The monitor pattern is a combination of single exclusive access and condition variables.}
  ,references={\cite[p. 399]{book:posa2}, \cite[p. 184]{book:cpj}}
]

\syncpattern [name={Condition Variables}
  ,category={Synchronization primitive}
  ,intent={Wait for a certain condition to become true.}
  ,status={Implemented language feature, possible library component}
  ,example={When a buffer is empty, consumers can wait on the \lstinline!is_not_empty! conditon variable. Producers will send a signal on this variable when a new item is available.}
  ,knownapps={Preconditions in SCOOP are effectively condition variables due to their wait semantics.}
  ,references={\cite[p. 298 and 306]{book:java-concurrency}}
]

\syncpattern [name={Synchronous Message Passing}
  ,category={Synchronization primitive}
  ,intent={Send a message from a sender to a receiver synchronously, where both have to wait until the operation has completed.}
  ,status={Possible library component}
  ,example={Make a flight reservation with the implicit guarantee that the booking system has received the message.}
  ,knownapps={Main synchronizaton mechanism in message passing systems.}
  ,references={\cite[p. 369]{book:cpj}}
]


\section {The SCOOP model}
\label {sec:scoop-model}
% Introduction, differences to Java etthirdc... (short)

SCOOP is an extension to the Eiffel language \todo{ref} that aims to make concurrent programming easier.
The basic idea is that every object can only be accessed by exactly one computational unit.
This unit is called processor, or handler of an object.

The keyword \lstinline!separate! is used to indicate that an object may be handled by a processor with respect to the handler for \lstinline!Current!.
Calls to a separate object (``separate calls'') then correspond to sending a message to the foreign processor.
There are two types of separate calls: synchronous and asynchronous.
If the called feature returns a result, the call is synchronous, which means that the current processor has to wait for the foreign processor to finish its task.
An asynchonous call happens when the feature is a command, i.e. not returning any result.
In that case, both processors can proceed concurrently.

A separate call is only allowed if the target of the call is ``controlled''.
Having an object controlled means that the user has exclusive access to that object - in that sense controlling an object corresponds a bit to locking in other languages.
In order to control an object it has to appear as a formal argument in the enclosing routine.

SCOOP guarantees that all messages sent by the current processor are handled in the foreign processor in the correct order.
A consequence of this and the exclusive access guarantee is that within a feature body, a separate object can be treated as if it were in a sequential program.
This is the reason why the SCOOP model is so simple: 
It allows reasoning about a feature body without the need to consider all possible interleavings of two parallel executions.

To create a new processor, one has to use the creation instruction on an object which is declared as separate.
The new processor is then initialized automatically, and the new object is handled by the new processor.

\todo{sortly describe wait conditions.}

There are many advantages to the SCOOP model, such as easier reasoning and absence of data races, but it also has some shortcomings.
One of them is the fact that it's a bit tedious to write SCOOP programs.
Due to the fact that every call to a separate object needs to be controlled, a user often has to write little helper functions that take a separate reference and just perform a single call on it.
It also has some performance problems, because encapsulating a separate call in a message may be expensive (especially for small functions like array access).
Additionally, a processor is currently implemented as an operating system thread, and creating them is an expensive operation that involves context switches.
The SCOOP model however encourages the creation of many processors, which is not ideal for performance reasons.

\section{Challenges in SCOOP}
\label{sec:scoop-challenges}
% This section describes recurring challenges in SCOOP and how to solve them...

\subsection{Object migration}
\label{sec:object-migration}

Passing data from one processor to another is often necessary when programming in SCOOP.
The most obvious example is the producer/consumer pattern, but it also applies to other situations where one just wants to provide some arguments to an asynchronous command.

\todo{what about expanded objects?}

There are basically three ways to safely pass reference objects from a sender to a receiver processor.
The first and easiest solution is to create data on its own, separate processor: 
\begin{lstlisting}
class SENDER feature
  send (a_receiver: separate RECEIVER)
      -- Invoke an asynchronous operation with
      -- an argument on `a_receiver'.
    local
      args: separate ANY
    do
      create args
      a_receiver.do_something (args)
    end
end

class RECEIVER feature 
  do_something (args: separate ANY)
      -- Perform some operation with `args'.
    do
      print (args)
    end
end
\end{lstlisting}
This approach is conceptually easy, but it has the drawback that it's not very efficient, especially when the argument object is very small.
We'll call this solution the Data Processor approach.

Another solution is to create the object on the same handler as the sender object:
\begin{lstlisting}
class SENDER feature
  send (a_receiver: separate RECEIVER)
      -- Invoke an asynchronous operation with
      -- an argument on `a_receiver'.
    local
      args: ANY
    do
      create args
      a_receiver.do_something (args)
    end
end

class RECEIVER feature 
  do_something (args: separate ANY)
      -- Perform some operation with `args'.
    do
      print (args)
    end
end
\end{lstlisting}
This solution (the Lock Passing approach) looks almost like the first one - the only change is a missing separate keyword.
However, it's semantics are radically different:

\begin{itemize}
 \item Due to the lock passing mechanism \todo{ref: piotr and eiffel docs} the feature \lstinline!do_something! is executed synchronously, i.e. the sender has to wait for it to finish.
 \item \lstinline!RECEIVER! can't access the argument object any more after the call \lstinline!do_something! is finished.
 This is because \lstinline!SENDER! will continue it's execution, and an attempt to lock the argument object again will probably result in starvation of the sender processor.
 \item Compared to the first approach, no new processor is created.
\end{itemize}

The last method makes use of a special SCOOP function called \lstinline!import!:
\begin{lstlisting}
class SENDER feature
  send (a_receiver: separate RECEIVER)
      -- Invoke an asynchronous operation with
      -- an argument on `a_receiver'.
    local
      args: ANY
    do
      create args
      a_receiver.receive_args (args)
      a_receiver.do_something
    end
end

class RECEIVER feature
  
  received: ANY
  
  receive_args (args: separate ANY)
      -- Receive some arguments
    do
      received := import (args)
    end

  do_something
      -- Perform some operation.
    do
      print (received)
    end
end
\end{lstlisting}
The \lstinline!import! feature copies its argument object, along with all non-separate references, to the local processor.
It is somewhat similar to \lstinline!{ANY}.deep_copy!, except that it doesn't clone separate references.

This import solution has several advantages.
There's no need for a new processor, and the receiver can keep the data and do the operation asynchronously.
The drawback however is that the data needs to be copied.
However, for small data items this is actually faster than creating a new thread.

Note that \lstinline!receive_args! is executed synchronously just like in the Lock Passing approach.
Therefore, to execute \lstinline!do_something! asynchronously, it has to be divided into an execution and argument receiving part.

The feature \lstinline!import! was first described in \todo{ref: Piotr}, but unfortunately it is not implemented in current SCOOP.
It is possible however to implement it manually with some user support.

\subsection{Processor communication}
\label{sec:processor-communication}
% Problem that two concurrent, active processors can't communicate. Example downloader task.
% Solution: a third, passive processor with a shared data structure.


It is often necessary that two threads need to communicate with each other.
One example would be a user interface with a background downloader task.
The user interface needs to be able to cancel the downloader, and the downloader needs to inform the GUI that it is finished.

In SCOOP this is not easily done.
Both processors are performing a long-running execution, which doesn't allow other processors to do separate calls on them.
Specifically, the GUI processor is in a main loop to receive input and repaint the window, whereas the downloader task is busy receiving chunks of data.

The solution to this kind of problem is to introduce a third processor which is ``passive'', meaning that it doesn't have a task to perform and only waits for incoming requests.
This third processor is known to the other two, ``active'' processors, and it contains all the attributes which are necessary for communication.
In our example this means that the ``passive'' processor takes care of an object with an \lstinline!is_cancelled! and \lstinline!is_finished! boolean flag.
The ``active'' processors then regularly need to check the status of these flags.

\subsection{Processor termination}
\label{sec:processor-termination}
% Problem: how to stop consumer waiting on empty buffer.
% Solution: Query is_stopped in shared buffer.

When an application needs to be shut down, it is necessary to stop any running threads.
Sometimes this can be done via the ``passive'' third processor as seen in Section \ref{sec:processor-communication}.
However, the active processor may be stuck in a wait condition.

One example of this could be the producer/consumer pattern, where a consumer is waiting for a buffer to become non-empty.
If all producers are already terminated, the consumer never gets the chance to break out of this wait condition and therefore it can't terminate successfully.

The solution is to add a query \lstinline!is_stop_requested! right inside the shared buffer, and to adapt to adapt the wait condition to include the stop request:

\begin{lstlisting}
class
  CONSUMER

feature -- Status report

  buffer: separate BUFFER
  
  last_item: INTEGER
  
  is_stopped: BOOLEAN
  
feature -- Basic operations
  
   start
      -- Start the main loop
    do
      from 
      until 
	is_stopped
      loop
	fetch (buffer)
	if not is_stopped then
	  -- do something
	end
      end
    end
  
feature -- Implementation

  fetch (buf: separate BUFFER)
      -- Get the next item from `buf'.
    require
      not buf.is_empty or buf.is_stop_requested
    do
      if buf.is_stop_requested then
	is_stopped := True
      else
	last_item := buf.item
	buf.remove
      end
    end
end
\end{lstlisting}


\section {Library}
\label{sec:library}

% The library is designed as a set of modules which simplify concurrent programming in SCOOP.
% Section ~\ref{sec:tutorial} explains how to use the library for commonly used patterns,
% while Section ~\ref{sec:modules} concentrates on the design of the library itself.
 
\subsection{Goals}
\label{sec:goals}

The goal of the library is to provide a set of tools that simplify programming in SCOOP.
Specifically, we want to provide implementations for common concurrency patterns like the worker pool.
The result should be a new SCOOP library similar to the standard concurrency libraries in Java \todoref or C\# \todoref.

The library was developed with the following design goals:

\begin{itemize}
 \item Avoid common SCOOP pitfalls, like deadlock problems, starvation of a processor, or unintentional lock passing.
 \item Shield the user from having to write ``small wrapper'' features, i.e. features that need to be written just to lock an object for a separate call.
 \item Reduce the overhead of thread creation, especially for concurrent programs that involve dealing with a lot of small separate objects.
\end{itemize}

\subsection{Concepts}

This section describes the core concepts of the library: Import and Separate Proxies.
The import concept deals with the problem of how to pass data from one processor to another.
It is usefol to achieve design goal \#3\todoref and to some extent \#1\todoref in Section \ref{sec:goals}.

The Separate Proxy is a pattern to hide separate references behind a proxy object.
It provides a solution to design goal \#2 \todoref.

\subsubsection{Import}
\label{sec:concepts:import}
% Describe how to use it, i.e. generic parameter and importer objects, and the fact that you can select between import and no import.

The import concept is a central part of the library.
It was developed to let users choose between two object passing strategies, namely the Data Processor \todoref and the Import \todoref approach.

The main class is the deferred \lstinline!CP_IMPORT_STRATEGY [G]!, which has the simple interface:
\lstinputlisting [firstline=7] {../../library/import/cp_import_strategy.e}

The class has two descendants: \lstinline!CP_NO_IMPORTER[G]! can be used for the Data Processor strategy. 
It just perform a reference copy of the object.
The class \lstinline!CP_IMPORTER [G]! on the other hand narrows the return type of \lstinline!import! to a non-separate \lstinline!G!, meaning that it actually performs an import.

As there's no general-purpose import feature available at the moment, a user has to implement his own feature for every class that needs to be imported.
Descendants of \lstinline!CP_IMPORTER! simplify this task and provide predefined implementations for some standard classes such as \lstinline!STRING!.
Those mechanisms are described in detail in Section \ref{sec:modules:import}.

\todo{elaborate more, explain less specific}
Other components of the library often use the import module via bounded genericity.
The class \lstinline!CP_QUEUE! for example has the ability to import objects, and it's class header looks like this:
\begin{lstlisting}
class CP_QUEUE
  [G, IMPORTER -> CP_IMPORT_STRATEGY [G] create default_create end]
feature
  ...
end
\end{lstlisting}
That way a user can decide on the precise semantics of the import strategy by just declaring the right type.



\subsubsection{Separate Proxy}

The separate proxy pattern provides a nice interface to a separate object, by providing a processor-local proxy which hides the separate reference.
It is applied to all classes in the library which are meant to be shared among processors, i.e. which are usually accessed through a separate reference.

The pattern consists of three classes:
\begin{enumerate} [label=(\arabic*)]
 \item\label{item:sep-proxy:first} The class for the actual separate object.
 \item\label{item:sep-proxy:second} A class that provides helper functions to access a separate object of type \ref{item:sep-proxy:first}, usually with the ending \lstinline!_UTILS!.
 \item\label{item:sep-proxy:third} A proxy class with the a similar interface as \ref{item:sep-proxy:first}, usually ending on \lstinline!_PROXY!.
    Using \ref{item:sep-proxy:second}, the proxy forwards all calls to an object of type \ref{item:sep-proxy:first}.
\end{enumerate}

\todo{A little diagram showing the class relations.}

It is possible to add a fourth, deferred class that just defines the interface for \ref{item:sep-proxy:first} and \ref{item:sep-proxy:third}.
However, there's an inconsistency: 
Any precondition in \ref{item:sep-proxy:first} which references \lstinline!Current! needs to be converted to a wait condition in \ref{item:sep-proxy:third} that references \ref{item:sep-proxy:first}.
Furthermore, not all features in the business class may be necessary in the proxy, and the proxy itself may add some more features such as compound actions.

Unfortunately this pattern cannot be fully turned into a module, because it's highly dependent on the precise interface of the business class.
There is however some support in the library: 
\lstinline!CP_PROXY! defines the creation procedure and the attributes \lstinline!subject! for a business class object and \lstinline!utils! for a helper object.

\todo{reference to appendix}

\subsection {Module overview}
\label{sec:module-overview}

The library consists of several modules which implement some of the patterns described in the overview (Section \ref{sec:pattern_overview}).

One of the most basic modules is the Import module in \dir{library/import}.
It implements the Import pattern \todoref and is at the same time one of the core concepts of the library.

The queue module in \dir{library/queue} implements the Prdocuer / Consumer pattern \todoref.
It depends upon the import module.

The process module in \dir{library/process} provides skeleton classes for objects with a main loop.
It provides implementations for the Active Object \todoref, Self Asynch \todoref and Timer: Periodic \todoref patterns.

The worker pool module in \dir{library/worker\_pool} implements the Worker Pool pattern \todoref.
It depends on the import, queue and process module.

The future module is located in \dir{library/executor} and \dir{library/promise}.
It provides an implementation for the Future \todoref as well as the Executor framework \todoref.

The Timer: InvokeLater pattern \todoref is implemented by a single class \lstinline!CP_DEFAULT_TASK! in \dir{libary/util}.

% \todo{Describe available modules, which patterns they're implementing, and which modules they depend upon.}


\section{Library modules}
\label{sec:modules}

\subsection{Import}
\label{sec:modules:import}

The import module substitutes the SCOOP \lstinline!import! feature, a built-in mechanism that is unfortunately not implemented at the moment.
% It consists of all classes in the directory \dir{library/import}. \\
The basic concepts and ideas behind the module are described in Section \ref{sec:concepts:import}.
This section only deals with the class \lstinline!CP_IMPORTER! and its descendants.

The class \lstinline!CP_IMPORTER [G]! has a single deferred feature \lstinline!import!.
It does not provide a generic import mechanism.
To write an importer for an arbitrary type, e.g. \lstinline!STRING!, a client needs to write a new class, inheriting from \lstinline!CP_IMPORTER [STRING]!, and implement the deferred feature.

Although the library has some predefined importers, e.g. for \lstinline!STRING!, writing an extra class for every user-defined type may quickly become tedious.
Therefore the library provides another way of using the import module with the class \lstinline!CP_IMPORTABLE!:

\begin{lstlisting}[captionpos=b, caption={The deferred class CP\_IMPORTABLE.}]
deferred class
  CP_IMPORTABLE

feature {CP_DYNAMIC_TYPE_IMPORTER} -- Initialization

  make_from_separate (other: separate like Current)
      -- Initialize `Current' with values from `other'.
    deferred
    end

end
\end{lstlisting}

Users can inherit from \lstinline!CP_IMPORTABLE! and define the import function right inside their class.

There are two predefined importers which can be used for \lstinline!CP_IMPORTABLE! objects: 
\begin{itemize}
 \item \lstinline!CP_DYNAMIC_TYPE_IMPORTER!
 \item \lstinline!CP_STATIC_TYPE_IMPORTER!
\end{itemize}
The latter uses constrained genericity to create an object of type \lstinline!G!.
The approach is pretty simple and fast but it has the drawback that the result type is always the static type \lstinline!G!, even if the argument to \lstinline!import! was of a subtype of \lstinline!G!.

The \lstinline!CP_DYNAMIC_TYPE_IMPORTER! on the other hand respects the dynamic type of its argument.
With the help of reflection it creates a new, uninitialized object of the correct type and then calls \lstinline!make_from_separate! to perform the initialization.
This introduces a new problem however with respect to void safety.

As opposed to the static type importer, the feature \lstinline!make_from_separate! doesn't need to be a creation procedure.
This in turn means that the compiler will not check if every attribute is correctly initialized.
% The object may not be correctly initialized if \lstinline!make_from_separate! is not declared as a creation procedure, as the new object will not be initialized.
It is therefore strongly advised to declare \lstinline!make_from_separate! as a creation procedure for every descendant of \lstinline!CP_IMPORTABLE!.

Another problem of the \lstinline!CP_DYNAMIC_TYPE_IMPORTER! is the invariant of an object.
There's a short time interval between the creation of an object (using reflection) and the call to \lstinline!{CP_IMPORTABLE}.make_from_separate! where the invariant is broken.
Due to this it is impossible to use classes with invariants in conjunction with the dynamic type importer.

In the future, there will hopefully exist an import routine natively supported by the SCOOP runtime.
In that case \lstinline!CP_IMPORTER! can be made effective and use the native import, and all its descendants will become obsolete.

\subsection{Queue}

The queue module provides the class \lstinline!CP_QUEUE! and some support classes that implement the Separate Proxy pattern \patternref{SP}.
The module can be used for the Producer / Consumer pattern \patternref{P/C}.
% It is mostly used to easily implement the producer / consumer pattern in a concurrent context.

The main challenge in the queue module is data migration, as described in Section \ref{sec:object-migration}.
Therefore the module makes heavy use of the import concept.
This means that, along with a generic argument for the data type, it is also necessary for clients to provide a \lstinline!CP_IMPORT_STRATEGY!.
The import strategy basically ``teaches'' the queue how to import a given object.

Internally, \lstinline!CP_QUEUE! uses an \lstinline!ARRAYED_LIST! to store its elements.

% It uses the separate proxy pattern, which means that it consists of three classes:
% 
% \begin{itemize}
%  \item \lstinline!CP_QUEUE!
%  \item \lstinline!CP_QUEUE_UTILS!
%  \item \lstinline!CP_QUEUE_PROXY!
% \end{itemize}
% 
% The first one provides the actual queue, but internally it just relies on \lstinline!ARRAYED_QUEUE!.
% The class \lstinline!CP_QUEUE_PROXY! can be used to access such a shared, separate queue without having to deal with separate references.
% 
% The interesting thing about this queue implementation is that it makes use of the import module.
% Along with the generic argument \lstinline!G! you can also provide an \lstinline!CP_IMPORT_STRATEGY [G]!.
% The import then happens automatically in both the queue and its proxy.

% \subsubsection{Producer / Consumer}
% 
% 
% The producer / consumer is a very popular pattern in concurrent programming, and it is a building block for other patterns like pipeline as well.
% The basic idea is to have a shared, concurrent buffer.
% Producer threads put new items into the buffer, whereas consumer threads remove items from this buffer.
% 
% \todo{BON-style diagram and ``migration graphics''}
% 
% In threaded systems, this buffer is usually accessed by several threads at the same time.
% Careful synchronization has to be enforced to ensure that the buffer remains in a consistent state.
% The data items on the other hand ``migrate'' from a producer to the buffer, and then to exactly one consumer.
% They therefore don't need to be thread-safe as long as the producer promises never to touch the item again.
% 
% In SCOOP things look a bit different however.
% Due to the exclusive access guarantee it is not necessary to establish a synchronization policy.
% The downside however is a loss of potential concurrency when a producer and a consumer accesses the buffer simultaneously, but this is a minor problem.
% 
% The main problem in SCOOP are the data items, especially if they are not of an expanded type.
% If the object is created on the producer processor, then the consumer needs to control the producer in order to get access to the object.
% This is clearly a situation that we want to avoid, because it couples the producer and consumer in a vicious hidden way, and the whole point of the producer / consumer pattern is to decouple the two.
% 
% A nice solution would be if it's somehow possible to migrate data items, like it's done in threaded languages.
% However, this is not possible with the current semantics of SCOOP, because an object always stays on the processor it was created on.
% 
% One approach to solve this problem is to create a new processor for every data item.
% This actually works, but it can be very slow, especially for a lot of small data items.
% There are two reasons for this:
% First, every SCOOP processor is mapped to an operating system thread, therefore creating a new processor involves creating a new thread which is an expensive operation.
% The second reason is the overhead of separate calls itself.
% This has to be paid every time the consumer wants to access the separate object.
% 
% Another problem of this approach is related to ease of programming.
% Dealing with a separate object can be very annoying, because you need to write small wrapper functions for every little feature call.
% 
% \begin{lstlisting}
% class
%   CONSUMER
% 
% feature -- Basic operations
%   
%   retrieve
%       -- Retrieve a string and print its length.
%     local
%       l_string: separate STRING
%     do
%       l_string := buffer_consume (a_queue)
%       print (string_count (l_string))
%     end
%     
% feature {NONE} -- Implementation
%   
%   buffer: separate BUFFER [STRING]
% 
%   buffer_consume (a_buffer: like buffer): separate STRING
%       -- An annoying small wrapper function for a buffer.
%     do
%       Result := a_buffer.item
%       a_buffer.remove
%     end
%     
%   string_count (a_string: separate STRING): INTEGER
%       -- An annoying small wrapper function for a string.
%     do
%       Result := a_string.count
%     end
% end
% \end{lstlisting}
% 
% 
% Due to these problems we decided to go for a different strategy: cloning objects.
% Using the import module it is possible to ``teach'' the shared buffer how to clone any user-defined object by just providing a generic argument.
% A first library for the producer / consumer pattern thus consisted of the class \lstinline!CP_QUEUE! and \lstinline!CP_IMPORT_STRATEGY!, along with some predefined importers.
% 
% The import trick solves the main problem of the producer / consumer, namely migrating objects from producer to consumer efficiently.
% However, producers and consumers still have to deal with a nasty separate reference (the shared buffer), and there's also the problem that a user of the library might forget to import objects on the consumer side.
% 
% To overcome this problem we implemented a non-separate proxy class which automatically deals with the separate reference and imports.
% This idea proved to be so successful that eventually it was turned into its own pattern: the separate proxy.
% 
% \todo {Bon-style graphics of CP\_QUEUE and related items.}


\subsection{Process}

The process module provides a set of classes that all implement a skeleton for a main loop with a deferred body.

The class \lstinline!CP_PROCESS! defines the interface.
It is a descendant of the class \lstinline!CP_STARTABLE!, which means that clients have a simple way to start a separate process using \lstinline!CP_STARTABLE_UTILS!.

Users need to implement the feature \lstinline!step!, which should contain the body of the loop.
The feature \lstinline!start! is used to start the loop, and it can be terminated by setting the attribute \lstinline!is_stopped! to \lstinline!True!.
% As the process module defines the skeleton for a main loop, users are just required to implement \lstinline!step!, which should contain the body of the loop.

\lstinline!CP_PROCESS! also introduces the two methods \lstinline!setup! and \lstinline!cleanup!.
They are called in the beginning or at the end of the main loop, and must be explicitly redefined by descendants if needed.

There are two techniques to implement the main loop itself.
The first technique, used by \lstinline!CP_CONTINUOUS_PROCESS!, is pretty staightforward:
\begin{lstlisting}
from setup
until is_stopped
loop
  step
end
\end{lstlisting}
This approach is very simple and fast. 
The problem however is that other processors never get a chance to access the \lstinline!CP_CONTINUOUS_PROCESS! unless the main loop is exited completely.
This class is a simple implementation of the Active Object pattern \patternref{AO}.

\lstinline!CP_INTERMITTENT_PROCESS! implements the other technique which is more interesting.
The basic idea is to perform only one iteration, and then ask some other processor to invoke the loop body again in \lstinline!Current!.
This ping-pong approach ensures that other processors get a chance to access and modify data in \lstinline!CP_INTERMITTENT_PROCESS! after each iteration.
In practice this is particularly useful to stop a \lstinline!CP_INTERMITTENT_PROCESS! from the outside.

\lstinline!CP_INTERMITTENT_PROCESS! implements the Asynchronous Self-Call pattern \patternref{ASC}.
The callback service is provided by the class \lstinline!CP_PACEMAKER!.
Every \lstinline!CP_INTERMITTENT_PROCESS! automatically creates an associated pacemaker.

The \lstinline!CP_PERIODIC_PROCESS! allows to add small delays between executions. 
It is a descendant of \lstinline!CP_INTERMITTENT_PROCESS! and an implementation of the Timer: Periodic pattern \patternref{TP}.
The class also introduces the simple command \lstinline!stop!, which can be used to stop the timer.


\subsection{Worker pool}
\label{sec:worker_pool} 

A worker pool is a set of threads that are ready to execute tasks.
The intention of the worker pool is to make use of parallelism while avoiding the overhead of expensive thread creation.

The main component of the worker pool is a shared buffer where clients can insert tasks to be executed.
A set of worker threads then continuously retrieve tasks from the buffer and execute them.
The worker pool module makes use of the queue module which provides the shared buffer.

The representation of a task to be executed varies between different languages.
In Java for instance a Runnable object is used, whereas in C\# the task is represented as a delegate.
SCOOP however has to deal with the problem of object migration, as described in Section \ref{sec:object-migration}.

If the task object is created on its own processor, as in the Data Processor approach, the performance advantage of the worker pool cancels out.
With the Lock Passing approach a task object will be executed on the processor that created the object, which kind of makes the worker pool useless (not to mention the risks of starvation if applied wrong).
This only leaves the import mechanism as a sensible solution.

The library supports two flavors of a worker pool.
The first and more basic one is to have a deferred class \lstinline!CP_WORKER! where clients can directly implement an operation.
The object submitted to the worker pool then corresponds to the arguments of the operation.

The second solution uses a special class to encapsulate an operation.
It is described in Section \ref{sec:executor}.

The basic worker pool module has three important classes:
\begin{itemize}
 \item \lstinline!CP_WORKER_POOL!
 \item \lstinline!CP_WORKER!
 \item \lstinline!CP_WORKER_FACTORY!
\end{itemize}

The \lstinline!CP_WORKER_POOL! provides the shared buffer and some additional functionality to adjust the pool size.
The type of the task object alongside its import strategy can be specified with generic arguments.
\lstinline!CP_WORKER_POOL! inherits from \lstinline!CP_QUEUE! and therefore uses the same import concept.

The deferred class \lstinline!CP_WORKER! corresponds to the worker thread in other languages.
Users need to implement the feature \lstinline!do_run!, which receives a task object and executes some operation on it.
The exact type of the task object depends on the generic arguments of \lstinline!CP_WORKER!, which must be the same as in \lstinline!CP_WORKER_POOL!.
The non-deferred part of \lstinline!CP_WORKER! is the main loop itself, which fetches a new task, calls \lstinline!do_run!, and checks if the worker needs to terminate.

The last class, \lstinline!CP_WORKER_FACTORY!, just provides a deferred factory function for a new worker.
The factory class is necessary because the exact type of \lstinline!CP_WORKER! is not known to the library in advance.
\lstinline!CP_WORKER_POOL! uses the factory to create new workers on demand.

% \subsubsection{Terminating workers}

An important functionality of a worker pool is to adjust the number of workers.
Increasing the worker count is easily done by just creating new instances of \lstinline!CP_WORKER!.
To decrease the amount of workers however the module needs to apply the processor termination technique described in Section \ref{sec:processor-termination}.


% However, decreasing the amount of workers is not that easy.
% 
% Java provides two builtin mechanisms to shut down a thread.
% You can either force it to stop, which immediately throws an exception in the thread \todo{ref}, or you can use the more collaborative interrupt mechanism \todo{ref}.
% The idea is that the thread will regularly check its interrupted flag and terminate on its own if requested.
% 
% The latter is also possible to do in SCOOP, except that there's no builtin interrupt flag.
% Instead a query \lstinline!is_stop_requested! in \lstinline!CP_WORKER_POOL! can be used.
% The main problem however are wait conditions.
% 
% In Java, blocking calls like \lstinline!wait()! and \lstinline!sleep()! may throw an \lstinline!InterruptedException! \todo{ref}.
% This avoids the problem that a thread may wait forever instead of shutting down, because all signaller threads have already terminated.
% Unfortunately, there's no such mechanism in SCOOP.
% It is possible however to work around this limitation by refining the wait condition:
% \begin{lstlisting}
% 
% class
%   CP_WORKER
%   
%   -- ...
%   
% feature -- Implementation
% 
%   fetch (pool: separate CP_WORKER_POOL)
%     require
%       not pool.is_empty or pool.is_stop_requested
%     do
%       if is_stop_requested then
% 	-- Stop the currrent worker.
%       else
% 	-- Grab the next item.
%       end
%     end
% 
% end
% \end{lstlisting}
% The additional \lstinline!if! statement is not very nice, but luckily it can be encapsulated completely in \lstinline!CP_WORKER!.
% 
% This code snippet is useful to break free of any wait condition if the requirements have changed.


The Separate Proxy pattern \patternref{SP} is applied to \lstinline!CP_WORKER_POOL! to make the handling of a separate worker pool object more convenient.

The basic worker pool module is very flexible.
It is for example possible to use it just as an advanced producer / consumer module where consumers are automatically created and destroyed.
The drawback however is that clients need to implement two classes, the worker and the factory, to make use of the module.
Section \ref{sec:executor} therefore introduces a more specialized version of the worker pool which can be used to execute arbitrary operations.



% \subsubsection{Arbitrary operations}
% \label{sec:arbitrary-operations}
% 
% So far the task of a worker is defined in a user-defined \lstinline!CP_WORKER! class, and the object submitted to the worker pool mostly contains data.
% The worker pool implementations in Java and C\# only accept Runnable (or delegate) objects.
% This enables arbitrary operations that can be executed by the worker threads.
% 
% The SCOOP version of the worker pool can be enhanced to act like the Java / C\# counterparts.
% To do that we need a class that represents an operation, and which can be moved across processor boundaries.
% 
% The agent classes in Eiffel (i.e. ROUTINE and descendants) may be used to represent operations, but they can't be easily imported.
% That's why we added a new, deferred class \lstinline!CP_TASK!.
% Users of the library can inherit from it and implement the feature \lstinline!run!.
% \todo {Tell about agent integration?}
% 
% Using this interface it is possible to have a predefined \lstinline!CP_TASK_WORKER! that just runs \lstinline!CP_TASK! objects.
% The associated \lstinline!CP_TASK_WORKER_POOL! implements the factory function and refines the raw \lstinline!CP_WORKER_POOL!.
% 
% The combination of these two classes is very close to the Java worker pool implementation.
% The only difference is that a \lstinline!CP_TASK! object needs to be imported, whereas a Java Runnable object doesn't.
% 
% \todo{Maybe merge subsections Arbitrary Operations and Promise into a new section that describes the ideas behind CP\_TASK?}

\subsection{Promise}
\label {sec:promise}

The promise module contains a set of classes which can be used to monitor the state of an asynchronous operation.
It is mostly used in conjunction with the executor or future module.

The main class is \lstinline!CP_PROMISE!, which defines queries like \lstinline!is_terminated! or \lstinline!is_exceptional!.
It also defines the interface to cancel an operation or to get the progress percentage (e.g. for a download task), but these queries need to be supported by the asynchronous operation as well.

The Separate Proxy \patternref{SP} is available for promise objects, because they are usually declared separate to the client.
In this case the pattern is implemented with four classes, i.e.
\begin{itemize}
 \item \lstinline!CP_PROMISE! defines a common interface,
 \item \lstinline!CP_SHARED_PROMISE! defines the actual separate object,
 \item \lstinline!CP_PROMISE_UTILS! has features to access a \lstinline!separate! \lstinline!CP_PROMISE! and
 \item \lstinline!CP_PROMISE_PROXY! implements the proxy object.
\end{itemize}

There's an important descendant, the \lstinline!CP_RESULT_PROMISE!, which is used for asynchronous operations returning a result.
It also has a set of associated classes that implement the Separate Proxy pattern.

The \lstinline!CP_RESULT_PROMISE! contains a query \lstinline!item! to retrieve the result as soon as it is available.
A distinguishing feature of this query is that it blocks if the result is not yet available.

The return type of \lstinline!item! depends on a generic argument.
To move the result back to the client the class makes use of the import concept.
This means that both \lstinline!CP_SHARED_RESULT_PROMISE! and \lstinline!CP_RESULT_PROMISE_PROXY! have an additional generic argument which defines the import strategy.

\subsection{Executor}
\label{sec:executor}

The executor module defines an interface for executing arbitrary tasks.
The implementation of the execution service may vary.
In most cases it is a worker pool, but it is also possible to use a single thread or to execute the task synchronously in the current thread.

The representation of a task object in Java is a Runnable object, or a delegate in C\#.
A SCOOP implementation also needs a class to represent an task, but with an important additional requirement: 
Its objects have to be importable.

The agent classes in Eiffel (i.e. \lstinline!ROUTINE! and descendants) may be used to represent operations, but they can't be easily imported.
Therefore we added a new, deferred class \lstinline!CP_TASK! to represent an importable asynchronous operation.
It also adds some additional functionality like exception handling or the ability to add a promise object (see Section \ref{sec:promise}).
To define a new task clients need to inherit from \lstinline!CP_DEFAULT_TASK! and implement the two features \lstinline!run! and \lstinline!make_from_separate!.

The interface to execute tasks is provided by the class \lstinline!CP_EXECUTOR!.
It defines the feature \lstinline!put! which takes a \lstinline!separate CP_TASK! object as its argument.
As an executor instance is usually placed on its own separate processor we applied the Separate Proxy pattern \patternref{SP} on \lstinline!CP_EXECUTOR!.

\todo {Tell about agent integration?}

The executor framework is pretty useless on its own, as it essentially consists of only two deferred classes.
Therefore it is shipped with a worker pool implementation.
The \lstinline!CP_TASK_WORKER_POOL! implements the executor interface and is itself a descendant of the more basic \lstinline!CP_WORKER_POOL!.
The associated \lstinline!CP_TASK_WORKER! then just fetches \lstinline!CP_TASK! objects and executes them.


\subsection{Futures}
\label{sec:futures}

The Future pattern \patternref{FUT} is used to perform a computation asynchronously.
Instead of computing a value straight away, the computation is wrapped into an object and the user receives a handle to retrieve the value as soon as it is ready.
This handle is often called Future, Promise or Delay.
In this section we'll use the term future for the whole pattern, and promise only refers to the handle.

The main advantage of the future pattern is that it allows to make use of parallelism in an easy way.
Users just have to spot computations which may run asynchronously and the future pattern then takes care of thread management, synchronization and result propagation.

The pattern consists of four building blocks:
\begin{itemize}
 \item The promise,
 \item the computation,
 \item the execution service,
 \item and a ``frontend'' object which takes a computation, submits it to the executor, and returns a promise.
\end{itemize}

The representation of the computation is a Callable object in Java and a delegate in C\#.
Our library uses the class \lstinline!CP_COMPUTATION! with the deferred feature \lstinline!computed!.
It is a descendant of \lstinline!CP_TASK! introduced in Section \ref{sec:executor}.

The promise object is defined by \lstinline!CP_PROMISE! and its descendants.
The detailed implementation is described in Section \ref{sec:promise}.

Due to the fact that \lstinline!CP_COMPUTATION! inherits from \lstinline!CP_TASK! we can just use the executor module (see Section \ref{sec:executor}) as the execution service for the future pattern.

The ``frontend'' part is provided by the two classes \lstinline!CP_EXECUTOR_PROXY! and \lstinline!CP_FUTURE_EXECUTOR_PROXY!.
This is an example for a proxy object where the responsability has been expanded:
Instead of just forwarding the computation object to the execution service, it also initializes the promise and returns it to the user.

The implementation of the future pattern hits two challenges:
\begin{itemize}
 \item Object Migration (see Section \ref{sec:object-migration}): Operations can't be easily moved from the client to an execution service.
 The same is also true for the result of a computation in the reverse direction.
 \item Processor Communication (see Section \ref{sec:processor-communication}): The promise object should neither be placed on the client processor nor on the executor service.
% The reason in both cases is that one processor may execute a main loop, which means the other processor never gets access to the promise object.
\end{itemize}

The first problem is already solved by the executor module. 
Just like a \lstinline!CP_TASK! object, a \lstinline!CP_COMPUTATION! is movable across processor boundaries.
To bring the result back to the client the promise module makes use of the import concept.

% (Section \ref{sec:concepts:import}.
% This means that both the Executor service and the Promise object have a generic argument to define the \lstinline!CP_IMPORT_STRATEGY!.
% The executor service always uses \lstinline!CP_DYNAMIC_TYPE_IMPORTER!, because \lstinline!CP_COMPUTATION! inherits from \lstinline!CP_IMPORTABLE!.
% The import strategy of the Promise object however is user-defined.
The second problem is more interesting however.
As we've seen in Section \ref{sec:processor-communication}, the promise object needs to be placed on a third processor.
However, starting a new processor for every computation introduces a huge overhead.

A better tradeoff would be to create one global processor which takes care of all promise objects.
This may introduce some contention if multiple futures are submitted at the same time, but we think that this is acceptable.

The global processor approach brings another problem though.
A promise object has two generic arguments for the return type and the import strategy.
As these arguments are not known in advance, and because SCOOP processor tags \cite[p. 90]{Nienaltowski07} are not implemented yet, it is impossible to create a promise object on this dedicated processor.

The solution is - surprisingly - the import concept.
We can create a ``template'' promise object with the correct types on the client processor, and then ask the global processor to import it.
That way the promise ends up on the correct processor.

\section{Evaluation}
\label {sec:evaluation}

To evaluate the library we implemented a small performance benchmark.
We implemented the Gaussian elimination algorithm \todoref in three different ways: sequentially, with SCOOP only, and using the future module \todoref from the library.
We chose to test the future pattern because it indirectly also measures many other parts of the library, like the worker pool or import mechanism.

We ran the tests with randomly generated matrices and each test was repeated 5 times.
A test matrix was square and its order always a power of two in the range from 32 to 1024.
Additionally, there was one more column for the result vector in the system of linear equations.
The test system was a quad-core AMD Phenom II X4 955 processor with 6 GB of RAM.

The results of the tests are shown in the table below:

\todo{calculate averages, insert table, write warning about special data size in Caption.}

From the results we can get several observations:

\begin{itemize}
 \item The raw SCOOP solution fails for the biggest matrix. 
 This is because it uses more than the maximum number of processors, and it's a known bug \todo{ref: https://docs.eiffel.com/book/solutions/scoop-implementation}.
 The library solution doesn't suffer from this problem because it's using a fixed amount of processors.
 \item The library is a lot faster than the raw SCOOP solution on large data sets.
 \item For smaller data sets, raw SCOOP beats the library.
 \item Sequential execution is a lot faster than SCOOP.
\end{itemize}

The last observation is probably the most fundamental.
The SCOOP runtime really needs to be improved in order to make it competitive to threaded systems, or even sequential ones.
Fortunately an improved version \todo{ref: Scott's thesis} is being developed at the time of writing.
It will be integrated into a future EiffelStudio release.

Another improvement which might be useful for the library is the Passive Processor concept \todoref.
This might be useful especially for the Future module, as both the worker pool and the promise object could be declared passive.

\section{Conclusion}
% Explain what is the impact of the selected patterns, and why they were selected.
% Say that import was developed to overcome language limitation
% Also: Future work


In this thesis we've worked out many methods that simplify concurrent programming in one way or another.
We've done a survey of concurrency patterns and described them in an extensive list.
This list can be used by anyone searching for a particular pattern.

From this list we selected some patterns which we thought to be especially useful.
The selection was based on the study of other concurrency libraries as well as some input from the CME research group \todoref.

The seleted patterns were then implemented and are now available as a new Eiffel library.
Besides the actual pattern implementations, this library also provides some workarounds for current SCOOP limitations, such as the missing import statement.

Performance measurements for the Future pattern implementation showed that the library is actually faster foe large data sets and uses less threads than the native SCOOP approach.

Finally we also describe some challenges when programming in SCOOP, and how they can be solved.
This is especially useful to programmers new to SCOOP, but which may have experience in concurrent programming with threads.

\subsection{Future work}

The library provides several opportunities for future work.

\begin{description}
 \item [More patterns] The library can be extended with further patterns.
 It may be useful to include Pipeline or Dataflow network.
 \item [Separate Proxies] It may be useful to apply the Separate Proxy patterns to some EiffelBase classes, such as \lstinline!ARRAYED_LIST!, \lstinline!HASH_TABLE! or \lstinline!ROUTINE!.
 \item [Separate Proxy Wizard] The creation of a separate proxy can be mostly automated with a wizard.
% \item [EiffelVision support] \todo{maybe?}
% \item [Agent integration] \todo{maybe?}
 \item [Concurrent Datastructures] Sometimes it may be useful to have truly concurrent data structures for performance reasons.
 Pure SCOOP doesn't allow to do this, but it can be done in a C library. 
\end{description}

A important step is to improve SCOOP itself.
There are various ways to do it, most of them related to performance improvements.

\begin{description}
 \item [Faster runtime] The SCOOP runtime needs to become faster. 
 This is currently being developed \todoref.
 \item [Native Import] A native SCOOP import feature is a good tool to deal with a lot of small objects.
 It will also make the library code and usage simpler, as the manual import workaround can be removed.
 \item [Passive Processors] The passive processors concept could be integrated into EiffelStudio.
 It can make a big performance improvement to situations where one needs to pass data from one processor to another.
 \item [Separate references] The handling of separate references should become more simple.
 At the moment a programmer is forced to write a lot of small, annoying features to perform separate calls.
 Some syntactic sugar would really be helpful.
\end{description}



\begin{appendix}

\section {API tutorial}
\label{sec:tutorial}

The library has several components which can be used for various tasks in concurrent programming.
This API tutorial therefore consists of three, mostly unrelated sections.
Each section describes a concurrency problem and how the library can be used to solve it.

All programming code used in this section originally comes from the example applications in the repository \cite{web:repository}.

\subsection{Producer / Consumer}

The producer / consumer example is pretty common in concurrent programming.
At its core is usually a shared buffer.
A producer can add items to the buffer, whereas a consumer removes items from the buffer.

The library class \lstinline!CP_QUEUE! can be used as a shared buffer.
If we want to use \lstinline!STRING! objects to be passed from producer to consumer, we have to declare the queue like this:

\begin{lstlisting}
class PRODUCER_CONSUMER feature

  make
      -- Launch producers and consumers.
    local
      queue: separate CP_QUEUE [STRING, CP_STRING_IMPORTER]
	-- ...
    do
      create queue.make_bounded (10)
	-- ...
    end
end
\end{lstlisting}

Note that there are two generic arguments:
\begin{itemize}
\item The first argument (\lstinline!STRING!) denotes the type of items in the queue.
\item The second argument (\lstinline!CP_STRING_IMPORTER!) defines the import strategy (see Section \ref{sec:concepts:import}).
  It teaches the queue how to import a string object.
\end{itemize}

% The import strategy is an important concept of the library.
% An import in the SCOOP context means that an object on a separate processor should be copied to the local processor.
% This is done recursively for any non-separate reference, i.e. for  \lstinline!STRING! you also have to copy the \lstinline!area! attribute.
% The import strategy can be used to tell a component if a given object should be imported, and if yes, how it is done.

In our example we decided to import every string object.
% In our example we're using the \lstinline!CP_STRING_IMPORTER! to import strings.
An alternative would be to use \lstinline!CP_NO_IMPORTER [STRING]! and deal with separate references instead.

The next step is to define the producer and consumer.

\begin{lstlisting}[language=OOSC2Eiffel, captionpos=b, caption={The producer class.}]
class
	PRODUCER

inherit
	CP_STARTABLE

create
	make

feature {NONE} -- Initialization

	make (a_queue: separate CP_QUEUE [STRING, CP_STRING_IMPORTER]; a_identifier: INTEGER; a_item_count: INTEGER)
			-- Initialization for `Current'.
		do
			identifier := a_identifier
			item_count := a_item_count
			create queue_wrapper.make (a_queue)
		end

	queue_wrapper: CP_QUEUE_PROXY [STRING, CP_STRING_IMPORTER]
			-- A wrapper object to a separate queue.

	identifier: INTEGER
			-- Identifier of `Current'.

	item_count: INTEGER
			-- Number of items to produce.

feature -- Basic operations

	start
			-- Produce `item_count' items.
		local
			i: INTEGER
			item: STRING
		do
			from
				i := 1
			until
				i > item_count
			loop
					-- Note that there's no need to declare `item' as 
					-- separate, because it will be imported anyway.
				item := "Producer: " + identifier.out + ": item " + i.out
				queue_wrapper.put (item)
				i := i + 1
			end
		end

end
\end{lstlisting}

You may notice three things in this example:

\begin{itemize}
 \item \lstinline!PRODUCER! inherits from \lstinline!CP_STARTABLE!.
 \item The \lstinline!PRODUCER! uses a \lstinline!CP_QUEUE_PROXY! instead of the \lstinline!CP_QUEUE!.
 \item The generated strings are not separate.
\end{itemize}

The classes \lstinline!CP_STARTABLE! and \lstinline!CP_STARTABLE_UTILS! are a useful combination.
They allow to start some operation on a separate object without the need for a specialized wrapper function.

\lstinline!CP_QUEUE_PROXY! is part of the Separate Proxy pattern \patternref{SP} (see Section \ref{sec:separate-proxy}).
It is useful to access a separate object without having to deal with separate references.

The fact that strings can be generated on the local processor is probably the most interesting observation.
Usually it is necessary when using SCOOP to create shared data on its own processor.
As we're using the import mechanism however this is not necessary and would be even wasteful.

% Usually this is not possible, because if the string is later passed to a consumer object, the latter needs to lock the producer in order to get access to the string.
% But in this case we instructed the queue object to import all string objects. During a call to \lstinline!queue_wrapper.put(item)! the following happens:
% \begin{itemize}
%  \item The producer waits until it gets exclusive access to the queue.
%  \item The separate call is executed. Because \lstinline!item! is a non-separate reference, the call is synchronous and lock passing happens.
%  \item The queue object imports the separate string, creating a local copy.
%  \item The separate call terminates, both processors can proceed autonomously.
% \end{itemize}
% Creating the strings on a separate processor is therefore unnecessary.

% The import trick avoids a lot of unnecessary thread creation:
% Instead of creating a new processor for every single produced item we just copy it, which is much faster for small objects.

The consumer is the same as the producer except for the feature \lstinline!start!:

\begin{lstlisting}[language=OOSC2Eiffel, captionpos=b, caption={The consumer class.}]
class
  CONSUMER
  
inherit
  CP_STARTABLE

  -- Initialization omitted...

feature -- Basic operations

	start
			-- Consume `item_count' items.
		local
			i: INTEGER
			item: STRING
		do
			from
				i := 1
			until
				i > item_count
			loop
				queue_wrapper.consume

				check attached queue_wrapper.last_consumed_item as l_item then

						-- Note that `item' is not declared as separate
					item := l_item
					print (item + " // Consumer " + identifier.out 
					  + ": item " + i.out + "%N")
				end
				i := i + 1
			end
		end
end
\end{lstlisting}

Again, there's no need to declare the consumed string as separate, thanks to the import mechanism.

% You might notice again that the consumed string is not declared as separate.
% This is again because of the import mechanism within \lstinline!CP_QUEUE!.

The only thing left now is to create and launch the producers and consumers in the main application.
Note that \lstinline!PRODUCER_CONSUMER! inherits from \lstinline!CP_STARTABLE_UTILS! such that it can use \lstinline!async_start! to start both the consumer and producer threads.

\begin{lstlisting}[language=OOSC2Eiffel, captionpos=b, caption={The producer / consumer application root class.}]
class
	PRODUCER_CONSUMER

inherit
	CP_STARTABLE_UTILS

create
	make

feature {NONE} -- Initialization

	make
			-- Launch the producer and consumers.
		local
			l_queue: separate CP_QUEUE [STRING, CP_STRING_IMPORTER]
			l_producer: separate PRODUCER
			l_consumer: separate CONSUMER
		do
			print ("%NStarting producer/consumer example. %N%N")

				-- Create a shared bounded queue for data exchange.
			create l_queue.make_bounded (queue_size)

				-- Create and launch the consumers.
			across 1 |..| consumer_count as i loop
				create l_consumer.make (l_queue, i.item, items_per_consumer)
				async_start (l_consumer)
			end

				-- Create and launch the producers.
			across 1 |..| producer_count as i loop
				create l_producer.make (l_queue, i.item, items_per_producer)
				async_start (l_producer)
			end
		end

feature -- Constants

	queue_size: INTEGER = 5
	producer_count: INTEGER = 10
	consumer_count: INTEGER = 10
	items_per_producer: INTEGER = 20
	items_per_consumer: INTEGER = 20

invariant
	equal_values: producer_count * items_per_producer = consumer_count * items_per_consumer
end
\end{lstlisting}

\subsection{Server thread}
\label{sec:echo-server}

In network server programming it is common to have a dedicated thread listening on a socket.
In a SCOOP environment it is not hard to create such a processor, but it is hard to stop it.
The main problem is that the server will run its own main loop, and other processors never get exclusive access to call a feature like \lstinline!stop!.

The library addresses this issue with \lstinline!CP_INTERMITTENT_PROCESS!.
The class defines a special main loop using the Asynchronous Self-Call pattern \patternref{ASC}.

To use \lstinline!CP_INTERMITTENT_PROCESS! you need to inherit from it and implement the deferred feature \lstinline!step!.
The following example defines a simple echo server that just listens on a socket and replies with the same string:

\begin{lstlisting}[language=OOSC2Eiffel, captionpos=b, caption={The echo server class.}]
class
	ECHO_SERVER

inherit

	CP_INTERMITTENT_PROCESS
		redefine
			cleanup
		end

create
	make

feature {NONE} -- Initialization

	make
			-- Initialization for `Current'.
		do
				-- Create the socket on the specified port.
			create socket.make_server_by_port (2000)
				-- Set an accept timeout.
			socket.set_accept_timeout (500)
				-- Enable the socket.
			socket.listen (5)
		end

feature -- Basic operations

	cleanup
			-- <Precursor>
		do
			socket.cleanup
		end

	stop
			-- Stop the current processor.
		do
			is_stopped := True
		end

		
	step
			-- <Precursor>
		local
			l_received: STRING
		do
				-- Accept a new message.
			socket.accept
			
				-- In case of an accept timeout `accepted' is Void.
			if attached socket.accepted as l_answer_socket then

					-- Read the message.
				l_answer_socket.read_line
				l_received := l_answer_socket.last_string

					-- Generate and send the answer.
				l_answer_socket.put_string (l_received)
				l_answer_socket.put_new_line
				l_answer_socket.close
			end
		end

feature {NONE} -- Implementation

	socket: NETWORK_STREAM_SOCKET
			-- The server network socket.

end
\end{lstlisting}
The accept timeout is important in this example.
It ensures that the server processor periodically breaks free of its wait condition while listening and therefore has a chance to finish the \lstinline!step! feature.

The echo server can be started with \lstinline!{STARTABLE_UTILS}.async_start! and stopped with the feature \lstinline!stop!.
Thanks to the special loop construct used in \lstinline!CP_INTERMITTENT_PROCESS!, stopping the echo server also works when called from another processor.

\subsection{Worker Pool and Futures}

This section describes how to use a worker pool for I/O tasks.
The application defines two operations on text files: reading a file and appending a string to a file.
The classes to represent these operations are \lstinline!FILE_APPENDER_TASK! and \lstinline!FILE_READER_TASK!.

The file reader task is implemented with the future module from the library.
The library has the class \lstinline!CP_COMPUTATION! which represents futures, i.e. asynchronous tasks that return a result.
The \lstinline!FILE_READER_TASK! therefore needs to inherit from \lstinline!CP_COMPUTATION!.

The file appender task doesn't return a result.
Therefore it has to inherit from \lstinline!CP_DEFAULT_TASK!.
This inheritance is necessary to be able to submit it to a worker pool later.

The two classes are shown in Listing \ref{code:file-tasks}.

\begin{lstlisting}[language=OOSC2Eiffel, label={code:file-tasks}, captionpos=b, caption={The file reader and appender classes.}]
class
	FILE_READER_TASK
inherit
	CP_COMPUTATION [STRING]

create
	make, make_from_separate

feature {NONE} -- Initialization

	make (a_path: STRING)
			-- Create a new task to read the content from `a_path'.
		do
			path := a_path
		end

feature {CP_DYNAMIC_TYPE_IMPORTER} -- Initialization

	make_from_separate (other: separate like Current)
			-- <Precursor>
		do
			create path.make_from_separate (other.path)
			promise := other.promise
		end

feature -- Access

	path: STRING
			-- The path of the file to read from.

feature -- Basic operations

	computed: STRING
			-- <Precursor>
		local
			l_file: PLAIN_TEXT_FILE
			l_content: STRING
		do
			create l_file.make_open_read (path)
			l_file.read_stream (l_file.count)
			Result := l_file.last_string
			l_file.close
		end
end

class
	FILE_APPENDER_TASK
inherit
	CP_DEFAULT_TASK

    -- Initialization similar to FILE_READER_TASK.
	
feature -- Access

	path: STRING
			-- The path of the file to write to.

	content: STRING
			-- The content to be written.

feature -- Basic operations

	run
			-- <Precursor>
		local
			l_file: PLAIN_TEXT_FILE
		do
			create l_file.make_open_append (path)
			l_file.put_string (content)
			l_file.close
		end
end
\end{lstlisting}

The main algorithm needs to be defined in \lstinline!computed! or \lstinline!run!, respectively.
Additionally, the feature \lstinline!make_from_separate! has to be defined.
This feature is required to import task objects from the client to the worker pool processor (see Section \ref{sec:concepts:import} for the import concept).

\lstinline!CP_EXECUTOR! defines an interface to submit and execute task objects.
The main implementation is \lstinline!CP_TASK_WORKER_POOL!.
The executor is shipped with a Separate Proxy \patternref{SP}, which means users can access it via \lstinline!CP_EXECUTOR_PROXY! or \lstinline!CP_FUTURE_EXECUTOR_PROXY!.
These two proxy classes also initialize the promise object, which is a handle to the asynchronous operation that can be used to wait for its termination or to retrieve the result when it's available.

Listing \ref{code:io-worker-pool} shows how a worker pool is used to submit file read and append tasks.

\begin{lstlisting}[language=OOSC2Eiffel, label={code:io-worker-pool}, captionpos=b, caption={Using a worker pool for futures and asynchronous tasks.}]
class
	IO_WORKER_POOL

create
	make

feature -- Constants

	path: STRING = "a.txt"
	
	hello_world: STRING = "Hello World%N"

feature {NONE} -- Initialization

	make
			-- Initialization for `Current'.
		do
				-- Create the worker pools.
			create worker_pool.make (100, 4)
			create executor.make (worker_pool)

				-- Run the example
			single_read_write

				-- Stop the executor. This is necessary such that 
				-- the application can terminate.
			executor.stop
		end

feature -- Basic operations

	single_read_write
			-- Perform a single write operation on a file.
		local
			write_task: FILE_APPENDER_TASK
			write_task_promise: CP_PROMISE_PROXY

			read_task: FILE_READER_TASK
			read_task_promise: CP_RESULT_PROMISE_PROXY [STRING, CP_STRING_IMPORTER]

			l_result: detachable STRING
		do
				-- Execute a file append task first.
			create write_task.make (path, hello_world)

				-- Submit the task and get a promise object.
			write_task_promise := executor.put_with_promise(write_task)

				-- Wait for the task to finish.
			write_task_promise.await_termination

				-- It is possible to check for IO exceptions.
			check no_exception: 
			  not write_task_promise.is_exceptional 
			end


				-- Now execute a read task.
			create read_task.make (path)

				-- Submit the task and get a promise.
			read_task_promise := executor.put_future (read_task)

				-- The promise can be used to retrieve the result.
				-- Note that the statement may block if the result
				-- is not ready yet.
			l_result := read_task_promise.item

				-- Check if the read-write cycle worked as expected.
% 			check correct_result: l_result ~  hello_world end
		end

feature {NONE} -- Implementation

	worker_pool: separate CP_TASK_WORKER_POOL
			-- The worker pool that executes the IO tasks.

	executor: CP_FUTURE_EXECUTOR_PROXY[STRING, CP_STRING_IMPORTER]
			-- The executor proxy for to submit tasks.

end
\end{lstlisting}

Submitting tasks to the executor and dealing with the asynchronous result is pretty straightforward.
Some calls to the promise object are blocking however, e.g. \lstinline!await_termination! or \lstinline!item!, if the asynchronous task has not finished yet.

The library ensures that no exception can escape the task object and crash the worker pool or the client code.
Clients can check if an exception happened with the query \lstinline!is_exceptional! on the promise object.
The exception trace, if any, is also available to the client with the query \lstinline!last_exception_trace!.

An important measure is to stop the executor when the application wants to terminate.
Otherwise the workers will continue to wait for incoming tasks, preventing the process to shut down.

\section{HowTo: Separate Proxy}


\todo{Adapt for appendix}

The following example shows a general recipe on how to create a separate proxy for the small class \lstinline!EXAMPLE!:

\begin{lstlisting}
deferred class
  EXAMPLE [G]

feature -- Status report

  is_available: BOOLEAN
      -- Is `item' available?
  
feature -- Access

  item: separate G
      -- Item in `Current'.
    require
      available: is_available
    deferred
    end

feature -- Element change

  put (a_item: separate G)
      -- Set `item' to `a_item'.
    deferred
    end

end
\end{lstlisting}

First we need to create the helper class.
This is done according to these rules:
 \begin{itemize}
  \item The name should be \lstinline!EXAMPLE_UTILS!.
  \item The generic arguments are the same as in \lstinline!EXAMPLE!
  \item Any feature to access the separate \lstinline!EXAMPLE! object can be prefixed by \lstinline!example_!.
  This helps to avoid name clashes if someone wants to inherit from \lstinline!EXAMPLE_UTILS!
  \item The first argument of each feature is \lstinline!example: separate EXAMPLE [G]!.
  All other arguments are the same as the ones in the corresponding feature in \lstinline!EXAMPLE!.
  \item Preconditions in \lstinline!EXAMPLE! should be rewritten as wait conditions with the same meaning in \lstinline!EXAMPLE_UTILS!.
  \item If there's a non-expanded return type to a feature, you can decide if it should be declared separate in \lstinline!EXAMPLE_UTILS! or if it should be imported.
 \end{itemize}

\begin{lstlisting}
class
  EXAMPLE_UTILS [G]
  
feature -- Access

  example_item (example: separate EXAMPLE [G]): separate G
      -- Get the item from `example'.
      -- May block if not yet available.
    require
      available: example.is_available
    do
      Result := example.item
    end

feature -- Element change
 
  example_put (example: separate EXAMPLE [G]; item: separate G)
      -- Put `item' into `example'.
    do
      example.put (item)
    end
end
\end{lstlisting}

In this example we also dropped the feature \lstinline!is_available!, because it's not considered to be important for separate clients.

The proxy class has also has some simple rules:

 \begin{itemize}
  \item The name should be \lstinline!EXAMPLE_PROXY!.
  \item The generic arguments are the same as in \lstinline!EXAMPLE!
  \item The class can inherit from \lstinline!CP_PROXY [EXAMPLE [G], EXAMPLE_UTILS [G]]! and declare \lstinline!make! as a creation procedure.
  \item The feature names and arguments are the same as in \lstinline!EXAMPLE!.
  \item Preconditions in \lstinline!EXAMPLE! are usually not present in \lstinline!EXAMPLE_PROXY!. They are wait conditions in \lstinline!EXAMPLE_UTILS! instead.
  \item Every feature implementation makes use of \lstinline!utils! to forward the request to the \lstinline!subject!.
 \end{itemize}

\begin{lstlisting}
class
  EXAMPLE_PROXY [G]

inherit
  CP_PROXY [EXAMPLE [G], EXAMPLE_UTILS [G]]

create
  make
  
feature -- Access

  item: separate G
      -- Item in the example object.
      -- May block if not yet available.
    do
      Result := utils.cell_item (subject)
    end

feature -- Element change

  put (a_item: separate G)
      -- Set `item' to `a_item'.
    do
      utils.cell_put (subject, a_item)
    end

end
\end{lstlisting}

\end{appendix}

\todos

\end{document}          

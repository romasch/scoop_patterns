\documentclass[a4paper,10pt]{article}
\usepackage[utf8]{inputenc}

\usepackage{pattern}
\usepackage{listings}
\usepackage{enumitem}
\usepackage{todo}
\usepackage{appendix}

\newcommand{\dir} [1] [] {#1} 
\newcommand{\todoref}{\todo{ref}}

% Title Page
\title{Concurrency Patterns in SCOOP}
\author{Roman Schmocker}


\begin{document}
\maketitle

\begin{abstract}

\end{abstract}

\tableofcontents

\section{Introduction}

Due to the advent of multicore processors, concurrent programming has become an important part in software engineering.
Dealing with parallelism isn't easy however.
There are many pitfalls, such as race conditions and deadlocks.

In practice programmers have learned to avoid tricky concurrency problems with the use of some well-known patterns \todo{is it ok to take introduction from project plan almost word for word?}.
These patterns are often shipped with the standard library of the language, such that users rarely have to implement them.

The Eiffel language \todo{ref} has a new concurrency extension called SCOOP, which stands for Simple Concurrent Object-Oriented Programming.
SCOOP simplifies concurrent programming a lot and eliminates one source of errors completely, namely race conditions \todo{ref}.
However, there is little experience on how to implement popular concurrency patterns, such as a worker pool, in SCOOP.

The goal of this thesis is therefore to implement a standard library for concurrency patterns in Eiffel.

Section \todoref introduces a list of concurrency patterns which we found and categorized by studying literature and the standard libraries.
A brief introduction to the SCOOP model is given in Section \todoref.
The focus of Section \todoref is to describe the library itself.
The first part contains an API tutorial, whereas the second part focuses on the design of the library itself.
Finally, Section \todoref describes some interesting patterns with special problems and the solutions to them.

\section {Pattern overview}
\label{sec:pattern_overview}
% Overview of all patterns, maybe tabular


\subsection{Data centered patterns}

\pattern [name={Producer / Consumer}
  ,category={Program structuring}
  ,intent={Provide a synchronized shared buffer. Producer threads put items into the buffer, and consumers remove items.}
  ,applicability={When participants should not know each other. Also applicable if there's no one-to-one relation between producers and consumers or when buffering is desired.}
  ,status={Implemented library component}
  ,example={A logger service, where many producers submit log messages to a buffer and a single consumer writes them to a file.}
  ,knownapps={Very widely used. E.g. logging, input processing, buffering web server requests.}
  ,relation={The worker pool uses this pattern to pass along tasks. Pipeline and Dataflow networks are several chained Producer / Consumer patterns.}
%  ,references={}
]

\pattern [name={Pipeline}
  ,category={Program structuring}
  ,intent={Process a stream of data in several independent stages.}
  ,applicability={When the input consist of a stream of data where several processing steps need to be performed.}
  ,status={Possible library component}
  ,example={An emailing system that applies a spam filter, database logging, and a virus scan to each incoming email.}
  ,knownapps={Messaging systems, multimedia streaming (receive - decode - display)}
  ,relation={The producer / consumer pattern is used between two stages. Pipeline is a special form of Dataflow Network.}
%  ,references={}
]

\pattern [name={Dataflow Network}
  ,category={Program structuring}
  ,intent={Process a stream of data in independent stages, with the option to branch and merge streams.}
  ,applicability={When the input consists of a stream of data which allows for parallel processing.}
  ,status={Possible library component}
  ,example={A video player application that internally has a file decoder stage, which splits the input in an audio and video part, which then gets further processed.}
  ,knownapps={The borealis engine \todoref.}
  ,relation={The pattern is related to Pipeline. However in Dataflow Network the data can be split by one stage and forwarded to two different stages and maybe merged again later.}
%  ,references={}
]

\pattern [name={Exchanger}
%   ,category={}
  ,intent={Exchange two objects between two threads atomically.}
  ,applicability={When the synchronization point and the atomicity is required.}
  ,status={Possible library component.}
  ,example={A logger with two buffers: One is used by clients to submit messages, the other is used by the logger to write messages. When the latter is empty and the former is full, an exchange happens.}
  ,knownapps={Hardly ever used.}
  ,relation={Similar to Handshake, except that data passes in both directions.}
%  ,references={}
]

\subsection{Task centered patterns}


\pattern [name={Worker Pool}
  ,category={Performance}
  ,intent={Avoid expensive thread cration by providing a set of threads that can execute arbitrary operations.}
  ,applicability={When there are a lot of small tasks that may be executed in parallel.}
  ,status={Implemented library component}
  ,example={A set of HTTP request handlers in a web server.}
  ,knownapps={Web servers.}
  ,relation={Producer / Consumer is used to pass along task objects. Worker Pool is usually an implementation of Executor framework.}
%  ,references={}
]

\pattern [ name={Future}
  ,category={Program structuring, Performance}
  ,intent={Run a task asynchronously, and fetch the result later.}
  ,applicability={When a computation may be run in parallel, but creating an extra thread is too expensive.}
  ,status={Implemented library component}
  ,example={A web browser which starts downloads tasks for image files in parallel to parsing and rendering the HTML file.}
  ,knownapps={In UI programming for long-running background tasks, or parallelization of certain numerical computations.}
  ,relation={Futures may be backed by worker pools that execute them.}
%  ,references={}
  ,comment={The wait-by-necessity semantics of SCOOP also corresponds to the Future pattern.}
]

\pattern [name={Executor framework}
  ,category={Program structuring}
  ,intent={Split task submission from task execution.}
  ,applicability={When the task execution strategy should be flexible, e.g. using a worker pool or creating a new thread per task.}
  ,status={Implemented library component.}
  ,example={The Java Executor interface, where descendants can decide wether a submitted Runnable object is executed in the current thread, in a new thread, or in a worker pool.}
  ,knownapps={Jave Executor interface, Microsoft TPL \todoref}
  ,relation={The worker pool is an implementation of an executor service.}
%  ,references={}
]

\pattern [name={Timer: Periodic}
  ,category={Program structuring}
  ,intent={Apply an operation repeatedly in regular intervals.}
  ,applicability={When an operation, which can be run in parallel to the applications main execution, needs to be applied repeadedly.}
  ,status={Implemented library component}
  ,example={An emailing application that checks for new messages every five seconds.}
  ,knownapps={Message polling, flushing buffers, repeated log write, heartbeat messages, cron jobs.}
  ,relation={Similar to Active Object, but it schedules just one operation repeatedly.}
%  ,references={}
]

\pattern [ name={Timer: Invoke later}
  ,category={Program structuring}
  ,intent={Invoke a certain operation at a later point in time.}
  ,applicability={When an operation can be run in parallel, at a later point in time.}
  ,status={Implemented library component}
  ,example={Send an email after a delay of one minute, in which the user can still press ``cancel''\todo{do we actually support cancellation?}}
  ,knownapps={``Grace periods'' to cancel an action, robotics control, alarm clocks.}
  ,relation={ - }
%  ,references={}
]

\subsection{Miscellaneous patterns}

\pattern [name={Active Object \todo{Examples etc...}}
  ,category={Program structuring}
  ,intent={Pair an object with its own thread. Feature calls are then transformed to asynchronous messages. 
  The object runs a main loop where it schedules requests from other processors and runs its own code.}
  ,applicability={When access to a shared resource can be guarded and scheduled by an object, or when an object may run its own main loop.}
  ,status={Implemented language mechanism. Implemented library component.}
%   ,example={}
%   ,knownapps={}
%   ,relation={}
]

\pattern [ name={Thread-local storage}
  ,category={Program structuring}
  ,intent={Provide private heap data for each thread.}
  ,applicability={When multiple threads run the same code, but each needs a different set of data, or if synchronization overhead for heap objects is undesirable.}
  ,status={Implemented language mechanism.}
  ,example={Store the last exception raised in the current thread.}
 ,knownapps={Java and C\# both have a class ThreadLocal<T>.}
 ,relation={ - }
%  ,references={}
 ,comment={Native support in SCOOP: use \lstinline!once(``THREAD'')! and non-separate return type.}
]

\pattern [name={Publish / Subscribe}
  ,category={Program structuring}
  ,intent={Provide a hook to subscribe to events. In the concurrent context there's often an intermediate broker which receives events from a publisher and forwards them to all subscribers.}
  ,applicability={When the publisher doesn't need to know the subscribers, and vice versa in the broker solution.}
  ,status={Possible library component}
  ,example={A GUI button has an event ``clicked'', where the application logic can provide a handler.}
  ,knownapps={Event driven programming, GUI frameworks like Java Swing or EiffelVision.}
%   ,relation={}
%  ,references={}
]

\pattern [name={Transactions}
  ,category={Program structuring}
  ,intent={Avoid a deadlock by ``reserving'' a set of objects one at a time. Abort if an object is already reserved by another thread.}
  ,applicability={When multiple operations need to be locked and no locking order can be established.}
  ,status={Possible library component}
  ,example={A banking application where two accounts need to be locked for a transfer.}
  ,knownapps={Two-phase locking in database systems.}
  ,relation={ - }
%  ,references={}
]

\pattern [name={Disruptor}
%   ,category={}
%   ,intent={}
%   ,applicability={}
%   ,status={}
%   ,example={}
%   ,knownapps={}
%   ,relation={}
]
% 
\pattern [name={Leader / Follower}
%   ,category={}
%   ,intent={}
%   ,applicability={}
%   ,status={}
%   ,example={}
%   ,knownapps={}
%   ,relation={}
]

\subsection{SCOOP patterns}

\pattern [ name={Import}
  ,category={Language limitation}
  ,intent={Copy an object structure from a separate processor to the local processor.}
  ,applicability={When it's cheaper to copy the object instead of creating it on a new processor.}
  ,status={Implemented library component; future language mechanism}
  ,example={Copy the HTTP request string from the network socket listener to a request handler, such that the listener can continue.}
  ,knownapps={The library makes heavy use of this pattern.}
  ,relation={ - }
%  ,references={}
]

\pattern [ name={Self Asynch}
  ,category={Program structuring}
  ,intent={Execute the body of a main loop, and then ask a separate processor to call back the loop body.}
  ,applicability={When a processor is running its own code, but others need to access data on it from time to time.}
  ,status={Implemented library component}
  ,example={A network socket listener that may be stopped by another process.}
  ,knownapps={The timer pattern and the echo server use self asynch.}
  ,relation={Similar to the Active Object pattern, but Self Asynch lets other processors manipulate its data directly.}
%  ,references={}
]

\pattern [ name={Separate Proxy}
  ,category={Program structuring}
  ,intent={Simplify access to a separate object by providing a processor-local proxy.}
  ,applicability={When a class is reusable (library code) and usually created on a separate processor.}
  ,status={Guideline}
  ,example={A shared queue which gets accessed by several thread. Each thread creates a processor-local proxy to avoid having to deal with a separate reference.}
  ,knownapps={Most classes in the library ship with a separate proxy.}
  ,relation={The separate proxy is a special version of the more general GoF \todo{ref} proxy pattern.}
%  ,references={}
]

\pattern [name={Full Asynchrony}
  ,category={Language limitation}
  ,intent={Perform an operation completely asynchronously.}
  ,applicability={When an operation can be run in parallel and there's no need to wait for a result.}
  ,status={Future language mechanism}
  ,example={A logger service where clients just want to send a log message without having to wait.}
  ,knownapps={A workaround exists in \todoref, but it's broken with current SCOOP.}
  ,relation={ - }
%  ,references={}
  , comment={Will be natively supported with the new runtime by Scott West.}
]

\pattern [name={Universal Call}
  ,category={Language limitation}
  ,intent={Provide a universal enclosing routine to perform a single separate call.}
  ,applicability={When calls to a separate object may be interleaved with calls from other processors.}
  ,status={Designed language mechanism}
  ,example={A separate queue where producers just insert new items.}
  ,knownapps={An implementation exists in \todoref, but it's broken with current SCOOP.}
  ,relation={The separate proxy is a workaround for the missing universal call (especially when no compound actions are defined).}
%  ,references={}
  , comment={The new language mechanism will probably be a statement like \lstinline!separate a as l_a then l_a.do_something end!.}
]


\pattern [name={EiffelVision support}
%   ,category={}
%   ,intent={}
%   ,applicability={}
%   ,status={}
%   ,example={}
%   ,knownapps={}
%   ,relation={}
%  ,references={}
]

\subsection{Synchronization primitives}


\syncpattern [name={Atomic Operations}
  ,category={Synchronization primitive}
  ,intent={Avoid the use of locks by using hardware-supported atomic operations.}
%   ,applicability={Shared memory systems}
  ,status={unsolved}
  ,example={A lock-free queue with the help of CompareAndSwap.}
  ,knownapps={Low-level primitive which is used to implement lock-free data structures or other synchronization primitives.}
%   ,relation={}
]
  
\syncpattern [name={Locks}
  ,category={Synchronization primitive}
  ,intent={An object where only one thread at a time succeeds in calling \lstinline!lock!, and others have to wait.}
%   ,applicability={}
  ,status={Possible library component}
  ,example={Provide exclusive access on a certain section of code.}
  ,knownapps={Low-level primitive which is often used to implement other synchronization primitives.}
%   ,relation={}
]

\syncpattern [name={TryLock}
  ,category={Synchronization primitive}
  ,intent={Try to acquire a lock, with the option to back off after a certain amount of time.}
%   ,applicability={}
  ,status={Possible library component}
  ,example={Database transactions may get aborted due to a timeout if they can't lock a resource after a certain amount of time.}
  ,knownapps={Applications with real-time requirements.}
%   ,relation={}
]

\syncpattern [name={Read / Write lock}
  ,category={Synchronization primitive}
  ,intent={Allow multiple concurrent readers, but provide exclusive access to a single writer.}
%   ,applicability={}
  ,status={Language limitation}
  ,example={An array with frequent concurrent reads can make use of a read / write lock.}
  ,knownapps={Shared, read-mostly data structures.}
%   ,relation={}
]

\syncpattern [name={Semaphore}
  ,category={Synchronization primitive}
  ,intent={Make sure that only a certain amount of threads can execute a certain code section.}
%   ,applicability={}
  ,status={Possible library component}
  ,example={The dining philosophers pattern, where at most (N-1) philosophers can eat.}
  ,knownapps={Can be used to implement other synchronization primitives.}
%   ,relation={Similar to a barrier.}
]

\syncpattern [ name={Single Exclusive Access}
  ,category={Synchronization Primitive}
  ,intent={Make sure that at most one thread has access to exactly one shared object or resource.}
%  ,applicability={Shared memory systems}
  ,status={Implemented language mechanism}
  ,example={A counter variable that shold only be incremented by one thread at a time to avoid lost updates.}
  ,knownapps={The Java ``synchronized'' block implements single exclusive access, as well as C\# ``lock''.}
%   ,relation={Usually implemented with some kind of locking mechanism.}
  ]

\syncpattern [ name={Multiple Exclusive Access}
  ,category={Synchronization Primitive}
  ,intent={Make sure that at most one thread has access to several shared objects or resources.}
%  ,applicability={Shared memory systems}
  ,status={Implemented language mechanism}
  ,example={A money transfer between two bank accounts.}
  ,knownapps={Databases can provide exclusive access over all data previously used in the same transaction.}
%   ,relation={It is possible to use nested single exclusive access to provide multiple exclusive access, but special care has to be taken with deadlocks.}
  ]

\syncpattern [name={Barrier}
  ,category={Synchronization primitive}
  ,intent={Provide a synchronization point where several threads have to meet before continuing.}
%   ,applicability={}
  ,status={Possible library component}
  ,example={If the computation of a matrix multiplication is split among threads, the barrier can be used to make sure that all threads finish before the result can be used}
  ,knownapps={Parallel matrix operations, parallel loop processing.}
%   ,relation={}
]

\syncpattern [name={Monitor}
  ,category={Synchronization primitive}
  ,intent={Ensure that only one thread has access to an object. The thread may also wait on a condition to become true or false.}
%   ,applicability={}
  ,status={Implemented language mechanism}
  ,example={A shared buffer with conditions \lstinline!is_empty! and \lstinline!is_full!.}
  ,knownapps={Java with a combination of \lstinline!synchronized!, \lstinline!wait ()! and \lstinline!notifiyAll()!}
  ,comment={The monitor pattern is a combination of single exclusive access and condition variables.}
]

\syncpattern [name={Condition Variables}
  ,category={Synchronization primitive}
  ,intent={Wait for a certain condition to become true.}
%   ,applicability={}
  ,status={Implemented language feature, possible library component}
  ,example={When a buffer is empty, a consumer can wait on the is\_not\_empty conditon variable. Producers will signal on this variable when a new item is available.}
  ,knownapps={Preconditions in SCOOP are effectively condition variables, due to their wait semantics.}
%   ,relation={}
]

\syncpattern [name={Synchronous message passing}
  ,category={Synchronization primitive}
  ,intent={Send a message from sender to receiver synchronously, where both must wait until the operation has completed.}
%  ,applicability={When the sender needs a delievery guarantee.}
  ,status={Possible library component}
  ,example={Make a flight reservation, with the implicit guarantee that the booking system has received the message.}
  ,knownapps={Main synchronizaton mechanism in message passing systems.}
%   ,relation={}
]


\section {The SCOOP model}
% Introduction, differences to Java etthirdc... (short)

SCOOP is an extension to the Eiffel language \todo{ref} that aims to make concurrent programming easier.
The basic idea is that every object can only be accessed by exactly one computational unit.
This unit is called processor, or handler of an object.

The keyword \lstinline!separate! is used to indicate that an object may be handled by a processor with respect to the handler for \lstinline!Current!.
Calls to a separate object (``separate calls'') then correspond to sending a message to the foreign processor.
There are two types of separate calls: synchronous and asynchronous.
If the called feature returns a result, the call is synchronous, which means that the current processor has to wait for the foreign processor to finish its task.
An asynchonous call happens when the feature is a command, i.e. not returning any result.
In that case, both processors can proceed concurrently.

A separate call is only allowed if the target of the call is ``controlled''.
Having an object controlled means that the user has exclusive access to that object - in that sense controlling an object corresponds a bit to locking in other languages.
In order to control an object it has to appear as a formal argument in the enclosing routine.

SCOOP guarantees that all messages sent by the current processor are handled in the foreign processor in the correct order.
A consequence of this and the exclusive access guarantee is that within a feature body, a separate object can be treated as if it were in a sequential program.
This is the reason why the SCOOP model is so simple: 
It allows reasoning about a feature body without the need to consider all possible interleavings of two parallel executions.

To create a new processor, one has to use the creation instruction on an object which is declared as separate.
The new processor is then initialized automatically, and the new object is handled by the new processor.

There are many advantages to the SCOOP model, such as easier reasoning and absence of data races, but it also has some shortcomings.
One of them is the fact that it's a bit tedious to write SCOOP programs.
Due to the fact that every call to a separate object needs to be controlled, a user often has to write little helper functions that take a separate reference and just perform a single call on it.
It also has some performance problems, because encapsulating a separate call in a message may be expensive (especially for small functions like array access).
Additionally, a processor is currently implemented as an operating system thread, and creating them is an expensive operation that involves context switches.
The SCOOP model however encourages the creation of many processors, which is not ideal for performance reasons.

\section{Challenges in SCOOP}
% This section describes recurring challenges in SCOOP and how to solve them...

\subsection{Object migration}
\label{sec:object-migration}

Passing data from one processor to another is often necessary when programming in SCOOP.
The most obvious example is the producer/consumer pattern, but it also applies to other situations where one just wants to provide some arguments to an asynchronous command.

\todo{what about expanded objects?}

There are basically three ways to safely pass reference objects from a sender to a receiver processor.
The first and easiest solution is to create data on its own, separate processor: 
\begin{lstlisting}
class SENDER feature
  send (a_receiver: separate RECEIVER)
      -- Invoke an asynchronous operation with
      -- an argument on `a_receiver'.
    local
      args: separate ANY
    do
      create args
      a_receiver.do_something (args)
    end
end

class RECEIVER feature 
  do_something (args: separate ANY)
      -- Perform some operation with `args'.
    do
      print (args)
    end
end
\end{lstlisting}
This approach is conceptually easy, but it has the drawback that it's not very efficient, especially when the argument object is very small.
We'll call this solution the Data Processor approach.

Another solution is to create the object on the same handler as the sender object:
\begin{lstlisting}
class SENDER feature
  send (a_receiver: separate RECEIVER)
      -- Invoke an asynchronous operation with
      -- an argument on `a_receiver'.
    local
      args: ANY
    do
      create args
      a_receiver.do_something (args)
    end
end

class RECEIVER feature 
  do_something (args: separate ANY)
      -- Perform some operation with `args'.
    do
      print (args)
    end
end
\end{lstlisting}
This solution (the Lock Passing approach) looks almost like the first one - the only change is a missing separate keyword.
However, it's semantics are radically different:

\begin{itemize}
 \item Due to the lock passing mechanism \todo{ref: piotr and eiffel docs} the feature \lstinline!do_something! is executed synchronously, i.e. the sender has to wait for it to finish.
 \item \lstinline!RECEIVER! can't access the argument object any more after the call \lstinline!do_something! is finished.
 This is because \lstinline!SENDER! will continue it's execution, and an attempt to lock the argument object again will probably result in starvation of the sender processor.
 \item Compared to the first approach, no new processor is created.
\end{itemize}

The last method makes use of a special SCOOP function called \lstinline!import!:
\begin{lstlisting}
class SENDER feature
  send (a_receiver: separate RECEIVER)
      -- Invoke an asynchronous operation with
      -- an argument on `a_receiver'.
    local
      args: ANY
    do
      create args
      a_receiver.receive_args (args)
      a_receiver.do_something
    end
end

class RECEIVER feature
  
  received: ANY
  
  receive_args (args: separate ANY)
      -- Receive some arguments
    do
      received := import (args)
    end

  do_something
      -- Perform some operation.
    do
      print (received)
    end
end
\end{lstlisting}
The \lstinline!import! feature copies its argument object, along with all non-separate references, to the local processor.
It is somewhat similar to \lstinline!{ANY}.deep_copy!, except that it doesn't clone separate references.

This import solution has several advantages.
There's no need for a new processor, and the receiver can keep the data and do the operation asynchronously.
The drawback however is that the data needs to be copied.
However, for small data items this is actually faster than creating a new thread.

Note that \lstinline!receive_args! is executed synchronously just like in the Lock Passing approach.
Therefore, to execute \lstinline!do_something! asynchronously, it has to be divided into an execution and argument receiving part.

The feature \lstinline!import! was first described in \todo{ref: Piotr}, but unfortunately it is not implemented in current SCOOP.
It is possible however to implement it manually with some user support.

\subsection{Processor communication}
\label{sec:processor-communication}
% Problem that two concurrent, active processors can't communicate. Example downloader task.
% Solution: a third, passive processor with a shared data structure.


It is often necessary that two threads need to communicate with each other.
One example would be a user interface with a background downloader task.
The user interface needs to be able to cancel the downloader, and the downloader needs to inform the GUI that it is finished.

In SCOOP this is not easily done.
Both processors are performing a long-running execution, which doesn't allow other processors to do separate calls on them.
Specifically, the GUI processor is in a main loop to receive input and repaint the window, whereas the downloader task is busy receiving chunks of data.

The solution to this kind of problem is to introduce a third processor which is ``passive'', meaning that it doesn't have a task to perform and only waits for incoming requests.
This third processor is known to the other two, ``active'' processors, and it contains all the attributes which are necessary for communication.
In our example this means that the ``passive'' processor takes care of an object with an \lstinline!is_cancelled! and \lstinline!is_finished! boolean flag.
The ``active'' processors then regularly need to check the status of these flags.

\subsection{Processor termination}
\label{sec:processor-termination}
% Problem: how to stop consumer waiting on empty buffer.
% Solution: Query is_stopped in shared buffer.

When an application needs to be shut down, it is necessary to stop any running threads.
Sometimes this can be done via the ``passive'' third processor as seen in Section \ref{sec:processor-communication}.
However, the active processor may be stuck in a wait condition.

One example of this could be the producer/consumer pattern, where a consumer is waiting for a buffer to become non-empty.
If all producers are already terminated, the consumer never gets the chance to break out of this wait condition and therefore it can't terminate successfully.

The solution is to add a query \lstinline!is_stop_requested! right inside the shared buffer, and to adapt to adapt the wait condition to include the stop request:

\begin{lstlisting}
class
  CONSUMER

feature -- Status report

  buffer: separate BUFFER
  
  last_item: INTEGER
  
  is_stopped: BOOLEAN
  
feature -- Basic operations
  
   start
      -- Start the main loop
    do
      from 
      until 
	is_stopped
      loop
	fetch (buffer)
	if not is_stopped then
	  -- do something
	end
      end
    end
  
feature -- Implementation

  fetch (buf: separate BUFFER)
      -- Get the next item from `buf'.
    require
      not buf.is_empty or buf.is_stop_requested
    do
      if buf.is_stop_requested then
	is_stopped := True
      else
	last_item := buf.item
	buf.remove
      end
    end
end
\end{lstlisting}
% class
%   BUFFER
%   
% feature -- Status report
%   
%   is_stop_requested: BOOLEAN
%       -- Do the consumers need to stop?
%       
%   is_empty: BOOLEAN
%       -- Is the buffer empty?
% 
% end
% \end{lstlisting}


\section {Library}

% The library is designed as a set of modules which simplify concurrent programming in SCOOP.
% Section ~\ref{sec:tutorial} explains how to use the library for commonly used patterns,
% while Section ~\ref{sec:modules} concentrates on the design of the library itself.
 
\subsection{Goals}
\label{sec:goals}

The goal of the library is to provide a set of tools that simplify programming in SCOOP.
Specifically, we want to provide implementations for common concurrency patterns like the worker pool.
The result should be a new SCOOP library similar to the standard concurrency libraries in Java \todoref or C\# \todoref.

The library was developed with the following design goals:

\begin{itemize}
 \item Avoid common SCOOP pitfalls, like deadlock problems, starvation of a processor, or unintentional lock passing.
 \item Shield the user from having to write ``small wrapper'' features, i.e. features that need to be written just to lock an object for a separate call.
 \item Reduce the overhead of thread creation, especially for concurrent programs that involve dealing with a lot of small separate objects.
\end{itemize}

\subsection{Concepts}

This section describes the core concepts of the library: Import and Separate Proxies.
The import concept deals with the problem of how to pass data from one processor to another.
It is usefol to achieve design goal \#3\todoref and to some extent \#1\todoref in Section \ref{sec:goals}.

The Separate Proxy is a pattern to hide separate references behind a proxy object.
It provides a solution to design goal \#2 \todoref.

\subsubsection{Import}
\label{sec:concepts:import}
% Describe how to use it, i.e. generic parameter and importer objects, and the fact that you can select between import and no import.

The import concept is a central part of the library.
It was developed to let users choose between two object passing strategies, namely the Data Processor \todoref and the Import \todoref approach.

The main class is the deferred \lstinline!CP_IMPORT_STRATEGY [G]!, which has the simple interface:
\lstinputlisting [firstline=7] {../../library/import/cp_import_strategy.e}

The class has two descendants: \lstinline!CP_NO_IMPORTER[G]! can be used for the Data Processor strategy. 
It just perform a reference copy of the object.
The class \lstinline!CP_IMPORTER [G]! on the other hand narrows the return type of \lstinline!import! to a non-separate \lstinline!G!, meaning that it actually performs an import.

As there's no general-purpose import feature available at the moment, a user has to implement his own feature for every class that needs to be imported.
Descendants of \lstinline!CP_IMPORTER! simplify this task and provide predefined implementations for some standard classes such as \lstinline!STRING!.
Those mechanisms are described in detail in Section \ref{sec:modules:import}.

\todo{elaborate more, explain less specific}
Other components of the library often use the import module via bounded genericity.
The class \lstinline!CP_QUEUE! for example has the ability to import objects, and it's class header looks like this:
\begin{lstlisting}
class CP_QUEUE
  [G, IMPORTER -> CP_IMPORT_STRATEGY [G] create default_create end]
feature
  ...
end
\end{lstlisting}
That way a user can decide on the precise semantics of the import strategy by just declaring the right type.



\subsubsection{Separate Proxy}

The separate proxy pattern provides a nice interface to a separate object, by providing a processor-local proxy which hides the separate reference.
It is applied to all classes in the library which are meant to be shared among processors, i.e. which are usually accessed through a separate reference.

The pattern consists of three classes:
\begin{enumerate} [label=(\arabic*)]
 \item\label{item:sep-proxy:first} The class for the actual separate object.
 \item\label{item:sep-proxy:second} A class that provides helper functions to access a separate object of type \ref{item:sep-proxy:first}, usually with the ending \lstinline!_UTILS!.
 \item\label{item:sep-proxy:third} A proxy class with the a similar interface as \ref{item:sep-proxy:first}, usually ending on \lstinline!_PROXY!.
    Using \ref{item:sep-proxy:second}, the proxy forwards all calls to an object of type \ref{item:sep-proxy:first}.
\end{enumerate}

\todo{A little diagram showing the class relations.}

It is possible to add a fourth, deferred class that just defines the interface for \ref{item:sep-proxy:first} and \ref{item:sep-proxy:third}.
However, there's an inconsistency: 
Any precondition in \ref{item:sep-proxy:first} which references \lstinline!Current! needs to be converted to a wait condition in \ref{item:sep-proxy:third} that references \ref{item:sep-proxy:first}.
Furthermore, not all features in the business class may be necessary in the proxy, and the proxy itself may add some more features such as compound actions.

Unfortunately this pattern cannot be fully turned into a module, because it's highly dependent on the precise interface of the business class.
There is however some support in the library: 
\lstinline!CP_PROXY! defines the creation procedure and the attributes \lstinline!subject! for a business class object and \lstinline!utils! for a helper object.

\todo{reference to appendix}

\subsection {Architecture}
\label{sec:modules}

The library consists of several modules which implement some of the patterns described in the overview (Section \ref{sec:pattern_overview}).

One of the most basic modules is the Import module in \dir{library/import}.
It implements the Import pattern \todoref and is at the same time one of the core concepts of the library.

The queue module in \dir{library/queue} implements the Prdocuer / Consumer pattern \todoref.
It depends upon the import module.

The process module in \dir{library/process} provides skeleton classes for objects with a main loop.
It provides implementations for the Active Object \todoref, Self Asynch \todoref and Timer: Periodic \todoref patterns.

The worker pool module in \dir{library/worker_pool} implements the Worker Pool pattern \todoref.
It depends on the import, queue and process module.

The future module is located in \dir{library/executor} and \dir{library/promise}.
It provides an implementation for the Future \todoref as well as the Executor framework \todoref.

The Timer: InvokeLater pattern \todoref is implemented by a single class \lstinline!CP_DEFAULT_TASK! in \dir{libary/util}.

% \todo{Describe available modules, which patterns they're implementing, and which modules they depend upon.}

\section{Library implementation}

\subsection{Import}
\label{sec:modules:import}

The import module implements the SCOOP \lstinline!import! feature, which should be built-in but isn't implemented at the moment.
It is one of the most important modules of the library.
It consists of all classes in the directory \dir{library/import}.

The basic concepts on how to use the module is described in Section \ref{sec:concepts:import}.
This section only describes the descendants of \lstinline!CP_IMPORTER!, which provide predefined import functions for some types and support for writing an import feature manually.

The deferred class \lstinline!CP_IMPORTER [G]! defines the basic interface for a manually defined import feature.
To define an import feature for an arbitrary type, e.g. \lstinline!STRING!, one has to inherit from \lstinline!CP_IMPORTER [STRING]! and provide an implementation for the deferred \lstinline!import! feature.

% The main class is the deferred \lstinline!CP_IMPORT_STRATEGY [G]!, which has the simple interface:
% \lstinputlisting [firstline=7] {../../library/import/cp_import_strategy.e}
% By implementing \lstinline!import!, descendants can decide if an object of type \lstinline!G! shall be imported and if yes, how it's done.
% 
% The class \lstinline!CP_NO_IMPORTER[G]! is the default class to disable import and just perform a reference copy.
% Every other importer can inherit from \lstinline!CP_IMPORTER!, which narrows the return type of \lstinline!import! to a non-separate \lstinline!G!.

Because writing an extra importer for every importable object may be tedious, there's a support class \lstinline!CP_IMPORTABLE!:

\lstinputlisting [firstline=7] {../../library/import/cp_importable.e}

That way an import function can be written right inside the class that needs to be imported.

There are two classes which can be used for \lstinline!CP_IMPORTABLE! objects: \lstinline!CP_DYNAMIC_TYPE_IMPORTER! and \lstinline!CP_STATIC_TYPE_IMPORTER!.
The latter uses bounded genericity to directly create an object of type \lstinline!G!.
This has the drawback however that the type is statically \lstinline!G!, even if the argument to \lstinline!import! was of a subtype of \lstinline!G!.

The \lstinline!CP_DYNAMIC_TYPE_IMPORTER! tries to avoid this problem by using reflection.
This introduces a new problem with respect to void safety however, as the new object will not be initialized.
Therefore it is strongly advised to declare \lstinline!make_from_separate! as a creation procedure for every descendant of \lstinline!CP_IMPORTABLE!.

Another problem of the \lstinline!CP_DYNAMIC_TYPE_IMPORTER! are the invariants of an object.
There's a short time interval between the creation of an object (using reflection) and the call to \lstinline!{CP_IMPORTABLE}.make_from_separate! where the invariants are broken.
Due to this, it is not possible to have invariants in a class that will be imported using the dynamic type importer.

In the future, there will hopefully exist an import routine natively supported by the SCOOP runtime.
In that case \lstinline!CP_IMPORTER! can be made effective and use the native import, and all its descendants will become obsolete.


% Other components of the library often use the import module via bounded genericity.
% The class \lstinline!CP_QUEUE! for example has the ability to import objects, and it's class header looks like this:
% \begin{lstlisting}
% class CP_QUEUE
%   [G, IMPORTER -> CP_IMPORT_STRATEGY [G] create default_create end]
% feature
%   ...
% end
% \end{lstlisting}
% That way a user can decide on the precise semantics of the import strategy by just declaring the right type.

\subsection{Queue}

The queue module is a simple queue implementation in class \lstinline!CP_QUEUE!.
It is mostly used to easily implement the producer / consumer pattern in a concurrent context.


The pattern itself is all about data migration as described in Section \ref{sec:object-migration}.
Therefore the queue module makes heavy use of the import mechanism.
This means that, along with a generic argument for the data type, it is also necessary to provide a \lstinline!CP_IMPORT_STRATEGY!.
The import strategy then basically ``teaches'' the queue how to import a given object.

Internally, \lstinline!CP_QUEUE! uses an \lstinline!ARRAYED_LIST! to store its elements.

As the \lstinline!CP_QUEUE! is intended to be used as a separate object, the module provides support classes that implement the Separate Proxy pattern.

% It uses the separate proxy pattern, which means that it consists of three classes:
% 
% \begin{itemize}
%  \item \lstinline!CP_QUEUE!
%  \item \lstinline!CP_QUEUE_UTILS!
%  \item \lstinline!CP_QUEUE_PROXY!
% \end{itemize}
% 
% The first one provides the actual queue, but internally it just relies on \lstinline!ARRAYED_QUEUE!.
% The class \lstinline!CP_QUEUE_PROXY! can be used to access such a shared, separate queue without having to deal with separate references.
% 
% The interesting thing about this queue implementation is that it makes use of the import module.
% Along with the generic argument \lstinline!G! you can also provide an \lstinline!CP_IMPORT_STRATEGY [G]!.
% The import then happens automatically in both the queue and its proxy.

% \subsubsection{Producer / Consumer}
% 
% 
% The producer / consumer is a very popular pattern in concurrent programming, and it is a building block for other patterns like pipeline as well.
% The basic idea is to have a shared, concurrent buffer.
% Producer threads put new items into the buffer, whereas consumer threads remove items from this buffer.
% 
% \todo{BON-style diagram and ``migration graphics''}
% 
% In threaded systems, this buffer is usually accessed by several threads at the same time.
% Careful synchronization has to be enforced to ensure that the buffer remains in a consistent state.
% The data items on the other hand ``migrate'' from a producer to the buffer, and then to exactly one consumer.
% They therefore don't need to be thread-safe as long as the producer promises never to touch the item again.
% 
% In SCOOP things look a bit different however.
% Due to the exclusive access guarantee it is not necessary to establish a synchronization policy.
% The downside however is a loss of potential concurrency when a producer and a consumer accesses the buffer simultaneously, but this is a minor problem.
% 
% The main problem in SCOOP are the data items, especially if they are not of an expanded type.
% If the object is created on the producer processor, then the consumer needs to control the producer in order to get access to the object.
% This is clearly a situation that we want to avoid, because it couples the producer and consumer in a vicious hidden way, and the whole point of the producer / consumer pattern is to decouple the two.
% 
% A nice solution would be if it's somehow possible to migrate data items, like it's done in threaded languages.
% However, this is not possible with the current semantics of SCOOP, because an object always stays on the processor it was created on.
% 
% One approach to solve this problem is to create a new processor for every data item.
% This actually works, but it can be very slow, especially for a lot of small data items.
% There are two reasons for this:
% First, every SCOOP processor is mapped to an operating system thread, therefore creating a new processor involves creating a new thread which is an expensive operation.
% The second reason is the overhead of separate calls itself.
% This has to be paid every time the consumer wants to access the separate object.
% 
% Another problem of this approach is related to ease of programming.
% Dealing with a separate object can be very annoying, because you need to write small wrapper functions for every little feature call.
% 
% \begin{lstlisting}
% class
%   CONSUMER
% 
% feature -- Basic operations
%   
%   retrieve
%       -- Retrieve a string and print its length.
%     local
%       l_string: separate STRING
%     do
%       l_string := buffer_consume (a_queue)
%       print (string_count (l_string))
%     end
%     
% feature {NONE} -- Implementation
%   
%   buffer: separate BUFFER [STRING]
% 
%   buffer_consume (a_buffer: like buffer): separate STRING
%       -- An annoying small wrapper function for a buffer.
%     do
%       Result := a_buffer.item
%       a_buffer.remove
%     end
%     
%   string_count (a_string: separate STRING): INTEGER
%       -- An annoying small wrapper function for a string.
%     do
%       Result := a_string.count
%     end
% end
% \end{lstlisting}
% 
% 
% Due to these problems we decided to go for a different strategy: cloning objects.
% Using the import module it is possible to ``teach'' the shared buffer how to clone any user-defined object by just providing a generic argument.
% A first library for the producer / consumer pattern thus consisted of the class \lstinline!CP_QUEUE! and \lstinline!CP_IMPORT_STRATEGY!, along with some predefined importers.
% 
% The import trick solves the main problem of the producer / consumer, namely migrating objects from producer to consumer efficiently.
% However, producers and consumers still have to deal with a nasty separate reference (the shared buffer), and there's also the problem that a user of the library might forget to import objects on the consumer side.
% 
% To overcome this problem we implemented a non-separate proxy class which automatically deals with the separate reference and imports.
% This idea proved to be so successful that eventually it was turned into its own pattern: the separate proxy.
% 
% \todo {Bon-style graphics of CP\_QUEUE and related items.}


\subsection{Process}

The process module provides a set of classes that all implement some sort of skeleton for a main loop.

The class \lstinline!CP_PROCESS! defines the interface.
It inherits from \lstinline!CP_STARTABLE!, such that clients have a simple way to start a separate process using \lstinline!CP_STARTABLE_UTILS!.

The feature \lstinline!start! is used to start the main loop.
As the process module defines the skeleton for a main loop, users are just required to implement \lstinline!step!, which should contain the body of the loop.
The loop can be exited by setting the attribute \lstinline!is_stopped! to \lstinline!True!.

\lstinline!CP_PROCESS! also introduces the two methods \lstinline!setup! and \lstinline!cleanup!.
They are called in the beginning or at the end of the main loop, and must be explicitly redefined by descendants if needed.

There are two different implementations for the main loop itself.
\lstinline!CP_CONTINUOUS_PROCESS! implements the main loop in a straightforward manner:
\begin{lstlisting}
from setup
until is_stopped
loop
  step
end
\end{lstlisting}
The advantage of this approach is its simplicity.
However, other processors never get a chance to access data which is handled by the processor of the \lstinline!CP_CONTINUOUS_PROCESS!, unless the main loop is exited completely.
This class is a simple implementation of the \lstinline!Active Object! pattern described in \todo{ref}.

The second implementation in \lstinline!CP_INTERMITTENT_PROCESS! is more interesting.
The basic idea is to perform only one iteration, and then tell another processor to inoke the loop body again in \lstinline!Current!.
This ping-pong approach ensures that after every iteration other processors may get a chance to access and modify data in \lstinline!CP_INTERMITTENT_PROCESS!.
In practice this is especially useful to stop a \lstinline!CP_INTERMITTENT_PROCESS! from the outside.

\lstinline!CP_INTERMITTENT_PROCESS! implements the Self Asynch pattern described in \todo{ref}.
The callback service is provided by the class \lstinline!CP_PACEMAKER!, and every \lstinline!CP_INTERMITTENT_PROCESS! automatically creates an associated pacemaker.

The \lstinline!CP_PERIODIC_PROCESS! is a refinement of the \lstinline!CP_INTERMITTENT_PROCESS!, which allows to add a small delay between executions.
It also adds a simple command \lstinline!stop!, which can be used to stop the process from the outside.
It is an implementation of the Timer: Periodic pattern in \todo{ref}.

\subsection{Worker pool}
\label{sec:worker_pool} 

% The worker pool module provides support classes to build a worker pool.
% It makes use of the queue module to store the work items.
% 
% The module consists of three classes:
% 
% \begin{itemize}
%  \item \lstinline!CP_WORKER!
%  \item \lstinline!CP_WORKER_POOL!
%  \item \lstinline!CP_WORKER_FACTORY!
% \end{itemize}
% 
% The last one is just an factory class to create the user-defined \lstinline!CP_WORKER! objects.
% 
% The deferred class \lstinline!CP_WORKER! has a predefined main loop, where the object first checks if it needs to terminate, then grabs a new work item, and processes it.
% The processing step is deferred and needs to be implemented by the user.
% 
% The \lstinline!CP_WORKER_POOL! is the central management instance.
% Its primary task is to accept new work items from clients, but it can also be used to adjust the number of workers in the pool and to terminate all workers.
% The \lstinline!CP_WORKER_POOL! also implements the separate proxy pattern, which means that clients should access it through \lstinline!CP_WORKER_POOL_PROXY!.
% 
% \subsection{Worker pool}

A worker pool is a set of threads that are ready to execute tasks.
The intention of the worker pool is to make use of parallelism but avoid the overhead of thread creation, which can be quite expensive especially for small tasks.

The main component of the worker pool is a shared buffer, where clients can insert tasks to be executed.
A worker thread will then repeatedly retrieve a task from the buffer and execute it.
The worker pool module makes use of the queue module, which provides the shared buffer.
  %producer / consumer pattern, with the library client as a producer and the worker threads as consumers.

%An important part of the worker pool is also the ability to increase or decrease the amount of worker threads, or to completely stop all worker threads such that the application can terminate.

The representation of a task to be executed varies between different languages.
In Java for instance a Runnable \todo{ref} object is used, whereas in C\# the task is represented as a delegate \todo{ref}.

In SCOOP there's the problem of object migration, as described in Section \ref{sec:object-migration}.
If the task object is created on its own processor, as in the Data Processor approach, you cancel out the performance gain you tried to achieve with the worker pool.
With the Lock Passing approach, a task object will be executed on the processor that created the object, which kind of makes the worker pool useless (not to mention the risks of processor starvation if applied wrong).
This only leaves the Import mechanism as a sensible solution.

In the library we support two flavors of a worker pool.
The first and more basic one is to have a deferred class \lstinline!CP_WORKER! where clients can directly implement an operation.
The object submitted to the worker pool then corresponds to the arguments of the operation.

The second solution uses a special task class to encapsulate an operation.
It is described in Section \ref{sec:arbitrary-operations}.

% However, there are two solutions to this problem: You can either use the import mechanism, or make the worker class deferred and let clients implement the task directly.
% In the library we support both approaches, although the latter can be used to implement the import solution.
% Section~\ref{sec:arbitrary-operations} shows how to do this.

The basic worker pool module has three main classes:
\begin{itemize}
 \item \lstinline!CP_WORKER_POOL!
 \item \lstinline!CP_WORKER!
 \item \lstinline!CP_WORKER_FACTORY!
\end{itemize}

The \lstinline!CP_WORKER_POOL! provides the shared buffer and some additional functionality to adjust the pool size.
The type of the task object alongside its import strategy can be specified with a generic argument.
\lstinline!CP_WORKER_POOL! inherits from \lstinline!CP_QUEUE! and therefore uses the same import mechanisms.

The deferred class \lstinline!CP_WORKER! corresponds to the worker thread in other languages.
Users need to implement the feature \lstinline!do_run!, which takes a task object and executes it.
The exact type of the task object depends on the generic arguments of \lstinline!CP_WORKER!, which should be the same as in \lstinline!CP_WORKER_POOL!.
The non-deferred part in \lstinline!CP_WORKER! is the main loop itself, which fetches a new task, calls \lstinline!do_run!, and checks if the worker needs to terminate.

The last class, \lstinline!CP_WORKER_FACTORY!, just provides a deferred factory function for a new worker.
This is necessary because the exact type of \lstinline!CP_WORKER! is not known to the library.
With a user-defined factory, the \lstinline!CP_WORKER_POOL! can create new workers on demand.




% \subsubsection{Terminating workers}

An important functionality of a worker pool is to adjust the number of workers.
Increasing the worker count is not a problem, as you can just create new \lstinline!CP_WORKER! instances using the factory.
To decrease the amount of workers, the module makes use of the processor termination technique described in Section \ref{sec:processor-termination}.


% However, decreasing the amount of workers is not that easy.
% 
% Java provides two builtin mechanisms to shut down a thread.
% You can either force it to stop, which immediately throws an exception in the thread \todo{ref}, or you can use the more collaborative interrupt mechanism \todo{ref}.
% The idea is that the thread will regularly check its interrupted flag and terminate on its own if requested.
% 
% The latter is also possible to do in SCOOP, except that there's no builtin interrupt flag.
% Instead a query \lstinline!is_stop_requested! in \lstinline!CP_WORKER_POOL! can be used.
% The main problem however are wait conditions.
% 
% In Java, blocking calls like \lstinline!wait()! and \lstinline!sleep()! may throw an \lstinline!InterruptedException! \todo{ref}.
% This avoids the problem that a thread may wait forever instead of shutting down, because all signaller threads have already terminated.
% Unfortunately, there's no such mechanism in SCOOP.
% It is possible however to work around this limitation by refining the wait condition:
% \begin{lstlisting}
% 
% class
%   CP_WORKER
%   
%   -- ...
%   
% feature -- Implementation
% 
%   fetch (pool: separate CP_WORKER_POOL)
%     require
%       not pool.is_empty or pool.is_stop_requested
%     do
%       if is_stop_requested then
% 	-- Stop the currrent worker.
%       else
% 	-- Grab the next item.
%       end
%     end
% 
% end
% \end{lstlisting}
% The additional \lstinline!if! statement is not very nice, but luckily it can be encapsulated completely in \lstinline!CP_WORKER!.
% 
% This code snippet is useful to break free of any wait condition if the requirements have changed.


The Separate Proxy pattern is applied to \lstinline!CP_WORKER_POOL! to simplify the use of a separate worker pool object.

The basic worker pool module allows for a very flexible use. 
It is for example possible to use it just as an advanced producer / consumer module where consumers are automatically created and destroyed.
The next section introduces a more advanced implementation of the worker pool which builds on the basic module.



\subsubsection{Arbitrary operations}
\label{sec:arbitrary-operations}

So far the task of a worker is defined in a user-defined \lstinline!CP_WORKER! class, and the object submitted to the worker pool mostly contains data.
The worker pool implementations in Java and C\# only accept Runnable (or delegate) objects.
This enables arbitrary operations that can be executed by the worker threads.

The SCOOP version of the worker pool can be enhanced to act like the Java / C\# counterparts.
To do that we need a class that represents an operation, and which can be moved across processor boundaries.

The agent classes in Eiffel (i.e. ROUTINE and descendants) may be used to represent operations, but they can't be easily imported.
That's why we added a new, deferred class \lstinline!CP_TASK!.
Users of the library can inherit from it and implement the feature \lstinline!run!.
\todo {Tell about agent integration?}

Using this interface it is possible to have a predefined \lstinline!CP_TASK_WORKER! that just runs \lstinline!CP_TASK! objects.
The associated \lstinline!CP_TASK_WORKER_POOL! implements the factory function and refines the raw \lstinline!CP_WORKER_POOL!.

The combination of these two classes is very close to the Java worker pool implementation.
The only difference is that a \lstinline!CP_TASK! object needs to be imported, whereas a Java Runnable object doesn't.


\todo{Maybe merge subsections Arbitrary Operations and Promise into a new section that describes the ideas behind CP_TASK?}

\subsection{Futures}

The future pattern is used to perform a computation asynchronously.
Instead of computing a value directly, the computation gets wrapped into an object and the user only receives a handle to retrieve the value when it's ready.
This handle is often called Future, Promise or Delay.
In this section we'll use the term Future for the whole pattern, and Promise only refers to the handle.

The main advantage of the future pattern is that it allows to make use of parallelism in an easy way.
A user just has to spot computations which may run asynchronously, and the future pattern takes care of thread management, synchronization and result propagation.

The future pattern consists of four building blocks:
\begin{itemize}
 \item The Promise object,
 \item the computation,
 \item the execution service,
 \item and a ``frontent'' object which takes a computation, submits it to the executor, and returns a Promise object.
\end{itemize}

The representation of the computation is a Callable object in Java and a delegate in C\#.
Our library uses the interface \lstinline!CP_COMPUTATION! with the deferred feature \lstinline!computed!.
It is a descendant of \lstinline!CP_TASK! introduced in Section \ref{sec:arbitrary-operations}.

The Promise object is defined by \lstinline!CP_PROMISE! and its descendants.
The detailed implementation is described in Section \ref{sec:promise}.

The execution service part of the Future pattern can vary.
In most cases it is a worker pool which executes the computation objects and updates the Promise with the correct result.
However, it is also possible to implement it as a single thread, or even to executing them synchronously in the current thread.

To take this variation into account, we added a new interface \lstinline!CP_EXECUTOR!.
The \lstinline!CP_TASK_WORKER_POOL! introduced in Section \ref{sec:worker_pool} implements this interface and thus can be used as an executor service for the Future pattern.
We also applied the Separate Proxy pattern on \lstinline!CP_EXECUTOR!, as it is mostly accessed through a separate reference.

\todo {Maybe highlight that EXECUTOR is not just for the future pattern, but a general implementation of the Executor pattern.}

The ``frontend'' part is implemented in the \lstinline!CP_EXECUTOR_PROXY!.
This is an example where the responsability of a proxy object has been expanded:
Instead of just forwarding the \lstinline!CP_COMPUTATION! to the execution service, it also creates the Promise object and returns it to the user.

The implementation of the future pattern in SCOOP hits two challenges:
\begin{itemize}
 \item Object Migration (see Section \ref{sec:object-migration}: Operations can't be easily moved from the client to an execution service.
 The same is also true for the result of a computation in the reverse direction.
 \item Processor Communication (see Section \ref{sec:processor-communication}: The promise object should neither be placed on the client processor nor on the executor service.
% The reason in both cases is that one processor may execute a main loop, which means the other processor never gets access to the promise object.
\end{itemize}

The solution to the first problem is, once again, the import concept \todoref.
% This means that both the Executor service and the Promise object have a generic argument to define the \lstinline!CP_IMPORT_STRATEGY!.
% The executor service always uses \lstinline!CP_DYNAMIC_TYPE_IMPORTER!, because \lstinline!CP_COMPUTATION! inherits from \lstinline!CP_IMPORTABLE!.
% The import strategy of the Promise object however is user-defined.
The second problem is more interesting however.
As we've seen in Section \ref{sec:processor-communication}, the Promise object needs to be placed on a third processor.

However, if we start a new processor for every computation, we introduce a huge overhead.

A better tradeoff would be to introduce one global processor, which takes care of all promise objects.
This may introduce contention if multiple futures are submitted, but we think that this is acceptable.

However, this approach brings another problem.
A promise object has two generic arguments for the return type and the import strategy.
As these arguments are not known in advance, and because SCOOP processor tags \todo{ref} are not implemented yet, it is not possible to create a promise object on this dedicated processor.

The solution is - surprisingly - the import module.
We can create a promise object with the correct types on the client processor, and then ask the global processor to import it.
This way the promise object finally ends up on the correct processor.

% In the library, the Promise object is provided by the class \lstinline!CP_PROMISE! and its descendants.
% The computation is represented with \lstinline!CP_COMPUTATION!, or \lstinline!CP_TASK! for operations that don't return a result.
% The execution service is the deferred class \lstinline!CP_EXECUTOR! and \lstinline!CP_TASK_WORKER_POOL! is the main implementation.
% 
% The separate proxy pattern is applied on \lstinline!CP_EXECUTOR! and \lstinline!CP_PROMISE!.
% Besides acting as a processor-local proxy to a separate \lstinline!CP_EXECUTOR!, 
% the classes \lstinline!CP_EXECUTOR_PROXY! and \lstinline!CP_FUTURE_EXECUTOR_PROXY! are also responsible to create Promise objects on the global processor.

\subsubsection{Promise}
\label {sec:promise}

The promise module contains a set of classes which can be used to monitor the state of an asynchronous operation.

The main class is \lstinline!CP_PROMISE!, which defines several queries like \lstinline!is_terminated! or \lstinline!is_exceptional!.
It also defines the interface to cancel a task or to get the progress percentage (e.g. for a download task), but these queries need to be supported by the task itself.

The Separate Proxy mechanism is available for promise objects, because they are usually declared separate to the client.
In this case the pattern is implemented with four classes, i.e.
\begin{itemize}
 \item \lstinline!CP_PROMISE! defines a common interface,
 \item \lstinline!CP_SHARED_PROMISE! defines the actual separate object,
 \item \lstinline!CP_PROMISE_UTILS! defines helper functions to access a \lstinline!separate CP_PROMISE! and
 \item \lstinline!CP_PROMISE_PROXY! is the proxy object.
\end{itemize}

There's an important descendant, the \lstinline!CP_RESULT_PROMISE!, which is used for asynchronous operations that return a result.
It also has a set of associated classes that implement the Separate Proxy pattern.

The \lstinline!CP_RESULT_PROMISE! contains a query \lstinline!item! to retrieve the result as soon as it's available.
A distinguishing feature of this query is that it blocks if the result is not yet available.
The return type of \lstinline!item! depends on a generic argument.
To migrate the result back to the client, this class makes use of the import module - i.e. the \lstinline!CP_SHARED_RESULT_PROMISE! and \lstinline!CP_RESULT_PROMISE_PROXY! both have an additional generic argument which defines the import strategy.

% The \lstinline!CP_RESULT_PROMISE! is used for the future pattern.

% \subsubsection{Executor}
% 
% The executor module can be used to execute varying operations.
% The main class is \lstinline!CP_EXECUTOR!, which provides facilities to execute a \lstinline!CP_TASK! objects.
% 
% \lstinline!CP_TASK! itself represents a user-defined operation.
% It can be imported across processor boundaries and provides exception handling.
% To define a new task a client needs to inherit from \lstinline!CP_DEFAULT_TASK! and implement the feature \lstinline!run! and \lstinline!make_from_separate!.
% 
% A special kind of task is \lstinline!CP_COMPUTATION!, which can also return a value.
% This is needed to implement the future pattern.
% 
% The \lstinline!CP_EXECUTOR! makes use of the separate proxy pattern.
% However, the proxy also enhances the raw \lstinline!CP_EXECUTOR! interface with the option to attach a \lstinline!CP_PROMISE! object to a task.
% This can be used to query the status of a task, await termination, or get the computed result back in case of a \lstinline!CP_COMPUTATION!
% All \lstinline!CP_PROMISE! objects are created on a dedicated processor to avoid unnecessary thread creation.
% 
% An important implementation of the \lstinline!CP_EXECUTOR! is the \lstinline!CP_TASK_WORKER_POOL!.
% As the name suggests, this is a worker pool where each worker repeatedly executes \lstinline!CP_TASK! objects.
% 
% \todo {More executor implementations?}

%\section {Individual patterns}
% Description of a few patterns


\section{Evaluation and Benchmarks}


\todo{uncomment}%
\section {API tutorial}
\label{sec:tutorial}

\subsubsection{Producer / Consumer}

The producer / consumer example is pretty common in concurrent programming.
At its core is usually a shared buffer.
A producer can add items to the buffer, whereas a consumer removes items from the buffer.

The library class \lstinline!CP_QUEUE! can be used as a shared buffer.
If we want to use \lstinline!STRING! objects to be passed from producer to consumer, we have to declare the queue like this:

\begin{lstlisting}
class PRODUCER_CONSUMER feature

  make
      -- Launch producers and consumers.
    local
      queue: separate CP_QUEUE [STRING, CP_STRING_IMPORTER]
	-- ...
    do
      create queue.make_bounded (10)
	-- ...
    end
end
\end{lstlisting}

Note that there are two generic arguments:
\begin{itemize}
\item The first argument (\lstinline!STRING!) denotes the type of items in the queue.
\item The second argument (\lstinline!CP_STRING_IMPORTER!) defines the import strategy.
\end{itemize}

The import strategy is an important concept of the library.
An import in the SCOOP context means that an object on a separate processor should be copied to the local processor.
This is done recursively for any non-separate reference, i.e. for  \lstinline!STRING! you also have to copy the \lstinline!area! attribute.
The import strategy can be used to tell a component if a given object should be imported, and if yes, how it is done.

In our example we're using the \lstinline!CP_STRING_IMPORTER! to import strings.
An alternative would be to use \lstinline!CP_NO_IMPORTER [STRING]! if we want to disable imports.

The next step we need to do is to define the producer and consumer.

\lstinputlisting [firstline=7] {../../examples/producer_consumer/producer.e}

You may notice three things in this example:

\begin{itemize}
 \item \lstinline!PRODUCER! inherits from \lstinline!CP_STARTABLE!.
 \item The \lstinline!PRODUCER! uses a \lstinline!CP_QUEUE_PROXY! instead of the \lstinline!CP_QUEUE!.
 \item The generated strings are not separate.
\end{itemize}

The classes \lstinline!CP_STARTABLE! and \lstinline!CP_STARTABLE_UTILS! are a useful combination.
They allow to start some operation on a separate object without the need for a specialized wrapper function.

Another nice utility is the \lstinline!CP_QUEUE_PROXY!.
It is part of a pattern which is often used throughout the library - the separate proxy \todo{add reference}.
Basically it allows to access a separate queue without the need to deal with separate references.

The really interesting thing however is that the producer can generate strings on its local processor.
Usually this is not possible, because if the string is later passed to a consumer object, the latter needs to lock the producer in order to get access to the string.
But in this case we instructed the queue object to import all string objects. During a call to \lstinline!queue_wrapper.put(item)! the following happens:
\begin{itemize}
 \item The producer waits until it gets exclusive access to the queue.
 \item The separate call is executed. Because \lstinline!item! is a non-separate reference, the call is synchronous and lock passing happens.
 \item The queue object imports the separate string, creating a local copy.
 \item The separate call terminates, both processors can proceed autonomously.
\end{itemize}
Creating the strings on a separate processor is therefore unnecessary.

The import trick avoids a lot of unnecessary thread creation:
Instead of creating a new processor for every single produced item we just copy it, which is much faster for small objects.

The consumer is basically the same as the producer, except for the feature \lstinline!start!:

\begin{lstlisting}
class
  CONSUMER
--...
  
	start
			-- Consume `item_count' items.
		local
			i: INTEGER
			item: STRING
		do
			from
				i := 1
			until
				i > item_count
			loop
				queue_wrapper.consume

				check attached queue_wrapper.last_consumed_item as l_item then

						-- Note that `item' is not declared as separate, because it has been
						-- imported automatically.
					item := l_item
					print (item + " // Consumer " + identifier.out + ": item " + i.out + "%N")
				end
				i := i + 1
			end
		end
end
\end{lstlisting}

You might notice again that the consumed string is not declared as separate.
This is again because of the import mechanism within \lstinline!CP_QUEUE!.

The last thing we need to do is to create and launch the producers and consumers in the main application:

\lstinputlisting [firstline=7] {../../examples/producer_consumer/producer_consumer.e}


\subsubsection{Server thread}

In networking you often need a dedicated thread that listens on a socket.
In a SCOOP environment it is not hard to create such a processor, but it is hard to stop it.
The main problem is that the server processor will run its own main loop, and other threads ususally never get exclusive access to call some feature \lstinline!stop!.

The library addresses this issue with the \lstinline!CP_INTERMITTENT_PROCESS!.
This class defines a special main loop, where other processors can access the object after every iteration.

To use \lstinline!CP_INTERMITTENT_PROCESS! you need to inherit from it and implement the deferred feature \lstinline!step!.
The following example defines a simple echo server that just listens and replies with the same string:

\lstinputlisting [firstline=7] {../../examples/echo_server/echo_server.e}

To break out of the main loop, you need to set \lstinline!is_stopped! to \lstinline!True!.
This can be done from within the \lstinline!CP_INTERMITTENT_PROCESS! or from a separate processor by calling \lstinline!stop!.

To start the echo server you can use \lstinline!CP_STARTABLE_UTILS! and the feature \lstinline!asynch_start!:

\lstinputlisting [firstline=7] {../../examples/echo_server/echo_application.e}

\subsubsection{Futures}

A future is a computation which may run asynchronously, possibly returning a result.
The future is a very popular pattern in other languages like Java.

The idea is that the computation is encapsulated in an object which may be executed by another thread.
If there's a result to the computation, the client thread also gets a token back to access the result at some point in the future.
This token is usually called ``Future'' or ``Promise''.

The library mechanism to support futures are the class hierarchies rooted at \lstinline!CP_TASK!, \lstinline!CP_BROKER! and \lstinline!CP_EXECUTOR!.

The \lstinline!CP_DEFAULT_TASK! class is used to define the operation.
To use it you need to inherit from it and implement \lstinline!run! and \lstinline!make_from_separate!.
The latter is needed because SCOOP doesn't allow shared access to objects, which is why a \lstinline!CP_TASK! needs to be imported from one processor to another.
If the operation is returning a result, it is necessary to inherit from \lstinline!CP_COMPUTATION! and implement \lstinline!computed!.

\lstinline!CP_EXECUTOR! is used to submit a task to be executed.
The main implementation is \lstinline!CP_TASK_WORKER_POOL!, which is using a worker pool to execute \lstinline!CP_TASK! objects.
The executor shoud be accessed using a local \lstinline!CP_EXECUTOR_PROXY! and \lstinline!CP_FUTURE_EXECUTOR_PROXY!, which simplify access to the \lstinline!separate CP_EXECUTOR! and also initialize the Promise object.

The \lstinline!CP_BROKER! serves as the token to store the status of the computation and a possible result.
It is created on a separate processor, such that both the client and the task object can access it without the risk of a deadlock.
If you use \lstinline!CP_EXECUTOR_PROXY!, you can submit a task and receive a token object by using \lstinline!put_with_broker!.

\todo{integrate example} 


\section{Conclusion}

\begin{appendix}
%\section{How-To: Separate Proxy}
\label{sec:howto-separate-proxy}

This appendix shows how to implement a separate proxy based on a small class \lstinline!EXAMPLE!.

\begin{lstlisting} [captionpos=b, caption={The example class (protégé) where the separate proxy should be applied.}]
class interface
  EXAMPLE [G]

feature -- Status report

  is_available: BOOLEAN
      -- Is `item' available?
  
feature -- Access

  item: separate G
      -- Item in `Current'.
    require
      available: is_available

feature -- Element change

  put (a_item: separate G)
      -- Set `item' to `a_item'.

end
\end{lstlisting}

First we need to create the helper class.
This is done according to these rules:
 \begin{itemize}
  \item The name should end in \lstinline!_UTILS!, i.e. \lstinline!EXAMPLE_UTILS!.
  \item The generic arguments are the same as in \lstinline!EXAMPLE!.
  \item Any feature to access the separate \lstinline!EXAMPLE! object should be prefixed with \lstinline!example_!.
  This helps to avoid name clashes if someone wants to inherit from \lstinline!EXAMPLE_UTILS!.
  \item The first argument of each feature is \lstinline!example: separate EXAMPLE [G]!.
  All other arguments are the same as the ones in the corresponding feature in \lstinline!EXAMPLE!.
  \item Preconditions in \lstinline!EXAMPLE! should be rewritten as wait conditions with the same meaning in \lstinline!EXAMPLE_UTILS!.
  \item If there's a non-expanded return type to a feature, you can decide if it should be declared separate in \lstinline!EXAMPLE_UTILS! or if it should be imported.
 \end{itemize}

\begin{lstlisting} [captionpos=b, caption={The helper class for a separate EXAMPLE.}]
class
  EXAMPLE_UTILS [G]
  
feature -- Access

  example_item (example: separate EXAMPLE [G]): separate G
      -- Get the item from `example'.
      -- May block if not yet available.
    require
      available: example.is_available
    do
      Result := example.item
    end

feature -- Element change
 
  example_put (example: separate EXAMPLE [G];
	    item: separate G)
      -- Put `item' into `example'.
    do
      example.put (item)
    end
end
\end{lstlisting}

In this example we also dropped the feature \lstinline!is_available!, because it's not considered to be important for separate clients.

The proxy class also has some simple rules:

 \begin{itemize}
  \item The name should be \lstinline!EXAMPLE_PROXY!.
  \item The generic arguments are the same as in \lstinline!EXAMPLE!.
  \item Inheriting from \lstinline!CP_PROXY [EXAMPLE [G], EXAMPLE_UTILS [G]]! is recommended.
  That way one can avoid having to write the creation procedure \lstinline!make!.
  \item The feature names and arguments are the same as in \lstinline!EXAMPLE!.
  \item Preconditions in \lstinline!EXAMPLE! are not present in \lstinline!EXAMPLE_PROXY!. 
  The class \lstinline!EXAMPLE_UTILS! defines them as wait conditions instead.
  \item Every feature body makes use of \lstinline!utils! to forward its requests to the \lstinline!subject!.
 \end{itemize}

\begin{lstlisting} [captionpos=b, caption={The proxy class for a separate EXAMPLE.}]
class
  EXAMPLE_PROXY [G]

inherit
  CP_PROXY [EXAMPLE [G], EXAMPLE_UTILS [G]]

create
  make
  
feature -- Access

  item: separate G
      -- Item in the example object.
      -- May block if not yet available.
    do
      Result := utils.example_item (subject)
    end

feature -- Element change

  put (a_item: separate G)
      -- Set `item' to `a_item'.
    do
      utils.example_put (subject, a_item)
    end

end
\end{lstlisting}

\end{appendix}

\todos


\end{document}          

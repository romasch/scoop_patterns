
\section {API tutorial}
\label{sec:tutorial}

\subsection{Producer / Consumer}

The producer / consumer example is pretty common in concurrent programming.
At its core is usually a shared buffer.
A producer can add items to the buffer, whereas a consumer removes items from the buffer.

The library class \lstinline!CP_QUEUE! can be used as a shared buffer.
If we want to use \lstinline!STRING! objects to be passed from producer to consumer, we have to declare the queue like this:

\begin{lstlisting}
class PRODUCER_CONSUMER feature

  make
      -- Launch producers and consumers.
    local
      queue: separate CP_QUEUE [STRING, CP_STRING_IMPORTER]
	-- ...
    do
      create queue.make_bounded (10)
	-- ...
    end
end
\end{lstlisting}

Note that there are two generic arguments:
\begin{itemize}
\item The first argument (\lstinline!STRING!) denotes the type of items in the queue.
\item The second argument (\lstinline!CP_STRING_IMPORTER!) defines the import strategy.
\end{itemize}

The import strategy is an important concept of the library.
An import in the SCOOP context means that an object on a separate processor should be copied to the local processor.
This is done recursively for any non-separate reference, i.e. for  \lstinline!STRING! you also have to copy the \lstinline!area! attribute.
The import strategy can be used to tell a component if a given object should be imported, and if yes, how it is done.

In our example we're using the \lstinline!CP_STRING_IMPORTER! to import strings.
An alternative would be to use \lstinline!CP_NO_IMPORTER [STRING]! if we want to disable imports.

The next step we need to do is to define the producer and consumer.

\lstinputlisting [firstline=7] {../../examples/producer_consumer/producer.e}

You may notice three things in this example:

\begin{itemize}
 \item \lstinline!PRODUCER! inherits from \lstinline!CP_STARTABLE!.
 \item The \lstinline!PRODUCER! uses a \lstinline!CP_QUEUE_PROXY! instead of the \lstinline!CP_QUEUE!.
 \item The generated strings are not separate.
\end{itemize}

The classes \lstinline!CP_STARTABLE! and \lstinline!CP_STARTABLE_UTILS! are a useful combination.
They allow to start some operation on a separate object without the need for a specialized wrapper function.

Another nice utility is the \lstinline!CP_QUEUE_PROXY!.
It is part of a pattern which is often used throughout the library - the separate proxy \todo{add reference}.
Basically it allows to access a separate queue without the need to deal with separate references.

The really interesting thing however is that the producer can generate strings on its local processor.
Usually this is not possible, because if the string is later passed to a consumer object, the latter needs to lock the producer in order to get access to the string.
But in this case we instructed the queue object to import all string objects. During a call to \lstinline!queue_wrapper.put(item)! the following happens:
\begin{itemize}
 \item The producer waits until it gets exclusive access to the queue.
 \item The separate call is executed. Because \lstinline!item! is a non-separate reference, the call is synchronous and lock passing happens.
 \item The queue object imports the separate string, creating a local copy.
 \item The separate call terminates, both processors can proceed autonomously.
\end{itemize}
Creating the strings on a separate processor is therefore unnecessary.

The import trick avoids a lot of unnecessary thread creation:
Instead of creating a new processor for every single produced item we just copy it, which is much faster for small objects.

The consumer is basically the same as the producer, except for the feature \lstinline!start!:

\begin{lstlisting}
class
  CONSUMER
--...
  
	start
			-- Consume `item_count' items.
		local
			i: INTEGER
			item: STRING
		do
			from
				i := 1
			until
				i > item_count
			loop
				queue_wrapper.consume

				check attached queue_wrapper.last_consumed_item as l_item then

						-- Note that `item' is not declared as separate, because it has been
						-- imported automatically.
					item := l_item
					print (item + " // Consumer " + identifier.out + ": item " + i.out + "%N")
				end
				i := i + 1
			end
		end
end
\end{lstlisting}

You might notice again that the consumed string is not declared as separate.
This is again because of the import mechanism within \lstinline!CP_QUEUE!.

The last thing we need to do is to create and launch the producers and consumers in the main application:

\lstinputlisting [firstline=7] {../../examples/producer_consumer/producer_consumer.e}


\subsection{Server thread}

In networking you often need a dedicated thread that listens on a socket.
In a SCOOP environment it is not hard to create such a processor, but it is hard to stop it.
The main problem is that the server processor will run its own main loop, and other threads ususally never get exclusive access to call some feature \lstinline!stop!.

The library addresses this issue with the \lstinline!CP_INTERMITTENT_PROCESS!.
This class defines a special main loop, where other processors can access the object after every iteration.

To use \lstinline!CP_INTERMITTENT_PROCESS! you need to inherit from it and implement the deferred feature \lstinline!step!.
The following example defines a simple echo server that just listens and replies with the same string:

\lstinputlisting [firstline=7] {../../examples/echo_server/echo_server.e}

To break out of the main loop, you need to set \lstinline!is_stopped! to \lstinline!True!.
This can be done from within the \lstinline!CP_INTERMITTENT_PROCESS! or from a separate processor by calling \lstinline!stop!.

To start the echo server you can use \lstinline!CP_STARTABLE_UTILS! and the feature \lstinline!asynch_start!:

\lstinputlisting [firstline=7] {../../examples/echo_server/echo_application.e}

\subsection{Futures}

A future is a computation which may run asynchronously, possibly returning a result.
The future is a very popular pattern in other languages like Java.

The idea is that the computation is encapsulated in an object which may be executed by another thread.
If there's a result to the computation, the client thread also gets a token back to access the result at some point in the future.
This token is usually called ``Future'' or ``Promise''.

The library mechanism to support futures are the class hierarchies rooted at \lstinline!CP_TASK!, \lstinline!CP_BROKER! and \lstinline!CP_EXECUTOR!.

The \lstinline!CP_DEFAULT_TASK! class is used to define the operation.
To use it you need to inherit from it and implement \lstinline!run! and \lstinline!make_from_separate!.
The latter is needed because SCOOP doesn't allow shared access to objects, which is why a \lstinline!CP_TASK! needs to be imported from one processor to another.
If the operation is returning a result, it is necessary to inherit from \lstinline!CP_COMPUTATION! and implement \lstinline!computed!.

\lstinline!CP_EXECUTOR! is used to submit a task to be executed.
The main implementation is \lstinline!CP_TASK_WORKER_POOL!, which is using a worker pool to execute \lstinline!CP_TASK! objects.
The executor shoud be accessed using a local \lstinline!CP_EXECUTOR_PROXY! and \lstinline!CP_FUTURE_EXECUTOR_PROXY!, which simplify access to the \lstinline!separate CP_EXECUTOR! and also initialize the Promise object.

The \lstinline!CP_BROKER! serves as the token to store the status of the computation and a possible result.
It is created on a separate processor, such that both the client and the task object can access it without the risk of a deadlock.
If you use \lstinline!CP_EXECUTOR_PROXY!, you can submit a task and receive a token object by using \lstinline!put_with_broker!.

\todo{integrate example} 

At the core of most GUI toolkits is an event dispatching thread (EDT).
The EDT is running an infinite loop where it listens for user input events.

Only the EDT is allowed to modify user interface objects.
Background threads therefore have to enqueue agents to be executed by the EDT if they want to update the GUI.

EiffelVision also follows this design with the event dispatching code defined in \lstinline!EV_APPLICATION! and the enqueuing feature \lstinline!do_once_on_idle! in the same class.
Due to this design one might think that combining SCOOP with EiffelVision is impossible.

This is not true however.
EiffelVision implements a variation of the Asynchronous Self-Call pattern \patternref{ASC}.
The body of the loop is defined by the feature \lstinline!{EV_APPLICATION_I}.process_event_queue!, whereas the actual loop is implemented in \lstinline!EV_APPLICATION_HANDLER!.
Therefore, a user interface object can subscribe to events from separate processors in the same manner as it can subscribe to events on the same processor.

In fact the SCOOP model makes GUI programming a lot easier.
In threaded languages programmers constantly have to worry that only the EDT manipulates user interface objects.
SCOOP however gives this guarantee for free.
This completely eliminates a source of randomly occurring errors which are usually very hard to find.

We'll show a small example application which can download a file in the background.
The application has a simple graphical user interface and uses the executor module from the library.
Some highlights of the example are event propagation and the cancellation mechanism.
The full source code can be found in \dir{examples/eiffelvision\_downloader} in the repository \cite{web:repository}.

Let's first look at the background task.
The class \lstinline!DOWNLOAD_TASK! shown in Listing \ref{code:download-task} defines the background download task.
It is a descendant of \lstinline!CP_DEFAULT_TASK! such that it can be used in conjunction with an executor.

\begin{lstlisting}[language=OOSC2Eiffel, label={code:download-task}, captionpos=b, caption={The background download task.}]
class
	DOWNLOAD_TASK

inherit
	CP_DEFAULT_TASK

    -- Initialization omitted.

feature -- Access

	url: STRING

feature -- Basic operations

	run
			-- <Precursor>
		local
			download_fragments: ARRAYED_LIST [STRING]
			http_downloader: detachable HTTP_PROTOCOL
			size: INTEGER
		do
			create download_fragments.make (50)
			create http_downloader.make (create {HTTP_URL}.make(url))

			from
					-- Start the download
				http_downloader.set_read_mode
				http_downloader.open
				http_downloader.initiate_transfer
				size := http_downloader.count
			until
					-- Terminate when the download is finished 
					-- or the user cancels the download manually.
				http_downloader.bytes_transferred = size 
				or attached promise as l_promise 
				and then is_promise_cancelled (l_promise)
			loop
					-- Receive a single packet.
				http_downloader.read
				if attached http_downloader.last_packet as l_packet then
					download_fragments.extend (l_packet)
				end
					-- Update the progress information in the UI.
				if attached promise as l_promise then
					promise_set_progress (l_promise, 
					  http_downloader.bytes_transferred / size)
				end
			end

				-- Discard result. A real application would
				-- probably write it to a file
			download_fragments.wipe_out
			http_downloader.close
		rescue
			if attached http_downloader as dl and then dl.is_open then
				dl.close
			end
		end
end
\end{lstlisting}

The code is structured such that the loop body only handles a small chunk of the total download.
This allows to check for a cancellation request in regular intervals, and to publish the current progress to the shared promise object.

The rescue clause at the end ensures that the connection is correctly closed.
Note that this is the only exception handling which needs to be done, and it's not necessary to ``catch'' it with \lstinline!retry!.
The user interface is still safe though because the exception gets caught later, transformed to an unsuccessful termination event, and the user interface will be informed automatically.

Another major component is the class which defines the main window.
It contains a lot of boring EiffelVision initialization code however.
Listing \ref{code:main-window} therefore only shows the event handling part.

\begin{lstlisting}[language=OOSC2Eiffel, label={code:main-window}, captionpos=b, caption={The main window.}]
class
	MAIN_WINDOW
inherit
	EV_TITLED_WINDOW

			-- Initialization and GUI elements omitted.

feature -- Access

	executor: CP_EXECUTOR_PROXY
			-- An executor to submit background tasks to.

	download_handle: detachable CP_PROMISE_PROXY
			-- A handle to a possible background download.

	formatter: FORMAT_DOUBLE
			-- A formatter for progress values.

feature -- Status report

	is_cancelling: BOOLEAN
			-- Is the download about to terminate?
			
feature {NONE} -- Button press events

	on_download_button_clicked
			-- Handler for clicks on download button.
		local
			downloader: DOWNLOAD_TASK
			l_promise: CP_PROMISE_PROXY
		do
			if not attached download_handle then

				create downloader.make (url_text.text)
				l_promise := executor.new_promise
				download_handle := l_promise

				l_promise.progress_change_event.subscribe (agent on_progress)
				l_promise.termination_event.subscribe (agent on_terminated)

				downloader.set_promise (l_promise.subject)
				executor.put (downloader)
			end
		end

	on_cancel_button_clicked
			-- Handler for clicks on cancel button.
		do
			if 
			   not is_cancelling and 
			   attached download_handle as l_download
			then
				l_download.cancel
				is_cancelling := True
				status_text.set_text ("Cancelling download...")
			end
		end

feature {NONE} -- Background download events

	on_terminated (is_successful: BOOLEAN)
			-- Handler for termination events from download task.
		do
			if not is_successful then
				result_text.set_text ("Download aborted.")
			elseif is_cancelling then
				result_text.set_text ("Download cancelled.")
				is_cancelling := False
			else
				result_text.set_text ("Download finished.")
			end

			status_text.set_text ("No download in progress.")
			download_handle := Void
		end

	on_progress (progress: DOUBLE)
			-- Handler for progress change events from download task.
		do
			if not is_cancelling then
				status_text.set_text ("Download progress:" +
				    formatter.formatted (progress * 100) + "%%")
			end
		end
end
\end{lstlisting}

The most interesting feature is \lstinline!on_download_button_clicked!.
It starts a background task and sets up all event handlers, such that the user interface can react to progress change or termination events.
The events are predefined in \lstinline!CP_PROMISE! and sent automatically to all subscribers whenever the associated task changes the progress value or terminates.

The \lstinline!on_terminated! handler function is used to receive an event that the background task has finished.
There are three cases that need to be distinguished: normal termination without cancellation, normal termination with cancellation, and exceptional termination due to an error.

The last component is the class \lstinline!DOWNLOAD_APPLICATION!.

\begin{lstlisting}[language=OOSC2Eiffel, label={code:download-application}, captionpos=b, caption={Download application root class.}]
class
	DOWNLOAD_APPLICATION

create make

feature {NONE} -- Initialization

	make
			-- Start the download application example.
		local
			app: EV_APPLICATION
			main_window: MAIN_WINDOW
			worker_pool: separate CP_TASK_WORKER_POOL
		do
				-- Create a worker pool which can be 
				-- used to execute background downloads.
			create worker_pool.make (10, 1)
			create executor.make (worker_pool)

				-- Create application and main window.
			create app
			create main_window.make (executor)

				-- Don't forget to tear down the worker pool at the end.
			app.destroy_actions.extend (agent executor.stop)

				-- Start the GUI.
			main_window.show
			app.launch
		end

feature -- Access

	executor: CP_EXECUTOR_PROXY
			-- The worker pool for background tasks.

end
\end{lstlisting}

The class just creates and starts the event loop, the worker pool and the main window.
The most interesting part is how the worker pool gets stopped.
As the statement \lstinline!app.launch! just kick-starts the event loop and then returns, we can't destroy the worker pool after this statement as in a sequential program.
It is therefore necessary to install a handler to stop the executor when the event loop terminates.

